\chapter{Conclusions}

Device driver synthesis is a radical alternative to traditional driver development. It has the potential to drastically reduce the development effort required to create device drivers and to improve their reliability.

This dissertation has presented the design and implementation of the Termite driver synthesis tool. Termite is the first tool to combine automatic game-based synthesis with conventional manual development, allowing the developer to work anywhere on the spectrum from full automation to verified manual development. It is also the first practical synthesis tool based on abstraction refinement. Finally, it is the first synthesis tool to support automated debugging of input specifications. 

In order to develop Termite, I created a rigorous game-based formalism of device driver synthesis inspired by Termite-1. This allowed me to apply pre-existing theory and results from the field of reactive synthesis. I created and evaluated a practical predicate-based abstraction refinement algorithm for solving games that that utilises a new type of refinement to reduce imprecision in the controllable predecessor calculation. To the best of my knowledge, this is the first such algorithm described in the literature. I addressed key performance bottlenecks involved in applying predicate abstraction in a game setting and demonstrated that the algorithm performs well on real-world reactive synthesis benchmarks in Section~\ref{sec:solving_eval}. Furthermore, I demonstrated that it outperforms the previous state of the art algorithm which was not based on predicate abstraction.

I also built a code generator and counterexample generator. These, along with the TSL language, compiler, and graphical interface built by Dr Leonid Ryzhyk form the Termite tool. I evaluated \termite in Chapter~\ref{ch:userguided} by synthesising drivers for several typical embedded devices. In all cases the performance and code size of the drivers was similar to, or better than, the hand written drivers and the synthesis algorithm completed within 11 minutes. The majority of each driver was synthesized fully automatically and the rest was manually written and verified by the tool. Once device operation was understood, developing the specifications was straightforward and the process was enhanced considerably by the graphical counterexample guided debugger.

Based on these experimental results, I consider Termite to be an important step towards truly practical device driver synthesis. Specifically, the scalable synthesis algorithm is able to efficiently handle real-world device specifications, while the user-guided approach reliably leads to high-quality code.

Further research is needed to solve the key remaining problems described in Section~\ref{s:limitations}, primarily the DMA problem, which poses the main obstacle to synthesis of more complex drivers, and the grey-box synthesis problem, which limits the degree of automation achieved by Termite. 

Additional research may explore ways to improve the quality of automatically generated code and thus further reduce the need for user involvement. This includes performance- and power-aware synthesis. Lastly, this approach may be used for the automatic synthesis of hardened device drivers, i.e., drivers that gracefully handle misbehaving devices~\cite{Kadav_RS_09}, by detecting behaviour that deviates from the specification.

