\chapter{A specification Language for Symbolic Games}
\label{ch:asl_ref}

We describe a simple specification language for symbolic games. This language is used to illustrate, in the simplest possible way, how Termite's specifications work. Furthermore, I have created a compiler for this language as an alternate frontend for Termite and it is available as part of Termite. As a result, all of the examples given in this chapter will compile and synthesize with Termite.

Fundamentally, symbolic transition systems are specified by defining how each state variable is updated on each transition, as a function of the other state variables and the label variables. We call these functions update functions and there is one update function for each state variable. 

Our simple language consists of four sections for defining a transition system:
\begin{itemize}
    \item The state variable declaration
    \item The label variable declaration
    \item The set of initial states
    \item Variable update functions, one for each state variable
\end{itemize}

Variables in both the state and label sections may be declared as one of three types:
\begin{itemize}
    \item Fixed width integer
    \item Boolean
    \item Enumeration
\end{itemize}

The transition relation is specified using nested case statements.

The BNF grammar is given below.

\subsection{Identifiers}

An identifier is a name of a variable or a constant declared as part of an enumeration.

\begin{bnflisting}
    <ident>    := <letter> (<letter> | <digit> | "_")*
\end{bnflisting}

\subsection{Numbers}

Numbers may be specified in hexadecimal, octal or decimal.

\begin{bnflisting}
    <number>    := "0x" <hexNumber>
                 | "0o" <octNumber>
                 | <decNumber>
\end{bnflisting}

\subsection{Types}

Our specification language supports three different types: Booleans, fixed width bit vectors, and enumerations.

\begin{bnflisting}
    <boolTyp>  := "bool"
    <intTyp>   := "uint" "<" <digit>* ">"
    <enumTyp>  := "{" identifier* "}"
\end{bnflisting}

\subsection{Variables}

Variables are declared by giving a name of the variable and its type.

\begin{bnflisting}
    <type>     := <boolTyp> | <intTyp> | <enumTyp>
    <varDecl>  := <identifier> ":" <type>
\end{bnflisting}

\subsection{Boolean Expressions}

Boolean expressions consist of boolean literals, value expressions (to follow) combined with the equality operator, and other boolean expressions combined with logical connectives.

\begin{bnflisting}
    <boolLit>  := "true" | "false"
    <binExpr>  := <boolLit> 
                | <valExpr> "==" <valExpr> 
                | <binExpr> "||" <binExpr> 
                | <binExpr> "&&" <binExpr> 
                | "!" <binExpr>
\end{bnflisting}

\subsection{Value Expressions}
Each value expression is a number, an identifier or a case statement. A case statement consists of a list of cases. Each case consists of a boolean condition followed by a value expression. Thus, case statements may contain further case statements. The outcome of a case statement is the value of the first case whose boolean expression evaluates to true.

\begin{bnflisting}
    <caseStmt> := "case" "{" 
                      (<binExpr> ":" <valExpr> ";")*
                  "}"
    <valExpr>  := <caseStmt> | <number> | <identifier>
\end{bnflisting}

\subsection{Variable updates}

Update functions are specified by assigning the next state value of a state variable to a value expression.

\begin{bnflisting}
    <update>   := <identifier> ":=" <valExpr>
\end{bnflisting}

\subsection{Transition System}

Transition systems are defined by giving the set of state variables, the set of label variables, the initial states and the variable update functions. The set of initial states is specified by the characteristic formula of the set.

\begin{bnflisting}
    <spec>     := "State"       <varDecl>* 
                  "Label"       <varDecl>* 
                  "Init"        <binExpr>
                  "Transitions" <update>*
\end{bnflisting}
