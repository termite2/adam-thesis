\chapter{Driver Synthesis as a Game}

In this chapter we formalise the device driver synthesis problem using games. We show that GR(1) games are sufficient to capture the properties of the drivers that we require. We also develop the driver synthesis controllable predecessor.

\section{The Players}

To formalise the driver synthesis problem using games, the first thing we must define is the players:

\begin{itemize}
    \item Player~1 is the driver. 
    \item Player~2 is the environment, which consists of the device to be controlled as well as the operating system.
\end{itemize}

\section{Device and OS specifications and synchronization}

A core concept of Termite is the separation of device and operating system specifications. Games so far have been defined to have a single state machine that contains all of the states of the system as well as all of the actions. In Termite, we decouple the description of the operating system from the device to be controlled and then combine these to form a state machine before synthesis.

Both the device and OS specifications are given as state machines. They are defined symbolically. Each specification has its own set of state variables that it must supply update functions for, however, one specification may refer to another's state variables in its state update functions. There is also a global set of label variables which each specification may also refer to in its state update functions. An overview of a system consisting of a device and OS specification as well as their interconnections is given in figure~\ref{fig:specs}.

As the figure shows, device and OS specifications are identical to modules that describe Mealy machines in a hardware description language such as Verilog or VHDL\@. They consist of some private state, represented by D type flip-flops (though the variables are not restricted to single bits) that is updated solely by logic within that module. Other modules can read, but not write to, this private state through the Mealy machine's output variables. The label, which is chosen by one of the players, appears as input to both of the modules and they can use it in the update functions for their private state.

To ensure reusability of specifications, specifications do not directly refer to each other's state variables. Instead, they access them through well defined interfaces and there is a device class mechanism to standardise the interfaces of both the device and operating system specification interfaces. We delay description of this mechanism until Chapter~\ref{c:userguided}. For now we assume that specifications can access each other's state variables without restriction.

\section{An Example System Specification}

We give an example system specification to illustrate its decomposition into separate OS and device specifications. We give the system both as graphical state machines and in ASL.

\subsection{State Machines}

Figure~\ref{fig:dev_spec} is our example device specification. The labelled circles are states. Arrows represent transitions between states. Solid arrows are controllable transitions whereas dashed arrows are uncontrollable, as described in Section~\ref{sec:conc_games}. Each transition is labelled by at least one event. Conceptually, these events trigger the transitions. All events for which there are no outgoing transitions do not change the state, i.e.\ they are self loops and are not shown in the interest of keeping the diagrams concise. The short arrow with no source state indicates the initial state. There may be more than one initial state, in which case the initial state is chosen non-deterministically.

\subsection{Device Specification}

We start with the device specification given in Figure~\ref{fig:dev_spec}. Our device is a hypothetical trivial UART-like device. Instead of sending bytes, it sends notifications, which carry no value. Its programming interface consists of an action, \code{devSendReq}, that requests that the device send a notification. In a real system this could be a write to a particular bit in a control register that triggers the device action. This action only has an effect in the \code{devIdle} state and it causes the device to transition to the \code{devSending} state. From there, when the \code{devSent} event happens, which corresponds to the device actually sending the notification, the device transitions back to the \code{devIdle} state. Additionally, it emits the \code{classSent} event. We will respond to the \code{classSent} event in the OS specification in the next section. From here, another notification may be requested by performing the \code{devSendReq} action again.

A fragment of the symbolic encoding of the device specification is given in Figure~\ref{fig:dev_spec_asl}. There is only one state variable, \code{devState}, that toggles between two values: \code{devIdle} and \code{devSending}. Additionally, there are three boolean label variabes: \code{controllable}, \code{devSendReq} and \code{devSent}. The \code{devSendReq} and \code{devSent} variables correspond to the actions in the provious paragraph. The \code{controllable} variable is an additional variable to distinguish between controllable and uncontrollable events. 

The update function for the \code{devState} variable (the only variable in the device specification) consists of a single case statement. If the system is currently in the \code{devIdle} state and the \code{devSendReq} label variable is true, and, importantly, the \code{controllable} label variable is also true, then the system transitions to the \code{devSending} state. The \code{controllable} variable had to be true because this was a controllable transition. Additionally if the system is in the \code{devSending} state and the \code{devSent} variable is true and the transition is not controllable then the system transitions to the \code{devIdle} state. If neither of these conditions are met, the device remains in the same state. 

On line~\ref{fig:dev_spec:l:output} we defined the \code{classSent} event to occur when the device is in state \code{devSending}, the \code{devSent} label is true and the transition is not controllable. This matches the state machine. The \code{define} statement is actually an m4~\cite{m4} macro that will be used to substitute the definition of \code{classSent} into the operating system specification later. This behaves much like a signal or continuous assignment in a hardware description language, but, to keep our specification language minimal we have not implemented signals.

Lastly, this device model may be seen as a Mealy machine as shown in Figure~\ref{fig:dev_spec_mealy}. There is a single register for storing the current state and this state is updated on each transition using combinational logic which additionally takes the label variables as input. This combinational logic also produces the output boolean variable \code{classSent}.

\begin{figure}
\centering
\begin{subfigure}[t]{0.47\textwidth}
\includegraphics[width=\linewidth]{diagrams/exampleSpecDevice.pdf}
\caption{Device specification}
\label{fig:dev_spec}
\end{subfigure}
\hfill
\begin{subfigure}[t]{0.47\textwidth}
\includegraphics[width=\linewidth]{diagrams/devMealy.pdf}
\caption{Device specification}
\label{fig:dev_spec_mealy}
\end{subfigure}
\par\bigskip
\begin{subfigure}[b]{\textwidth}
\begin{asllisting}
State
devState : {devIdle, devSending};

Label
devSendReq   : bool;
devSent      : bool;
controllable : bool;

Init
devState == devIdle

Transitions
devState := case {
    devState == devIdle && devSendReq && controllable  : devSending;
    devState == devSending && devSent && !controllable : devIdle;
    true                                               : devState;
}

define(classSent, (devState == devSending && devSent && !controllable)) (*@\label{fig:dev_spec:l:output}@*)
\end{asllisting}
\caption{Symbolic device specification}
\label{fig:dev_spec_asl}
\end{subfigure}
\end{figure}

\subsection{OS specification}

The operating system specification, given in Figure~\ref{fig:os_spec} is like a test harness for our driver. It specifies which requests the driver may receive (e.g.\ to send a notification) and how it must respond (e.g.\ by eventually sending the requested notification). It is simply another state machine, like the device specification, but it may also have goals.

The goal state, state \code{O3}, is the state that a correct driver will eventually force the OS specification to be in. It is possible to specify multiple goal states, in which case the driver must force execution into any goal state.

Our OS state machine starts off in the \code{osIdle} state. From there, if a \code{classSent} event is triggered by the device state machine, then the device must have sent a notification without the OS ever having requested one. This is an error, so the OS state machine transitions to the \code{osError} state. There is no way that the OS could have known that this event happened as it does not monitor the device notification output, so, the OS specification does not represent the way that the OS behaves in reality. These events that do not correspond to real interactions but are used to enforce correctness are called \emph{virtual events}. Responding to virtual events in this way is how the OS spec guarantees correctness of the system. 

There is nothing special about the error state, however, we have constructed the state machine in such a way that the error state is a dead end and it is not possible to reach the goal from that state.

If a \code{devSndReq} event happens, the OS transitions into the \code{devRequested} state. From there, when a \code{classSent} event happens, the OS specification transitions into the goal state as this time the \code{classSent} event was expected. 

We can see from the OS specification that it is the job of the driver to ensure that a \code{devSendReq} event is eventually followed by \code{classSent} event. However, a \code{classSent} event can only be issued by the device, not the driver. The driver can, however, force the device to issue a \code{classSent} event indirectly in response to a \code{devSendReq} event. 

In general, it is the job of the driver synthesis algorithm to figure out how to force certain events to happen at the right time (as specified by the OS state machine) by using information from the device state machine. 

The symbolic specification is given in Figure~\ref{fig:os_spec_asl}. After seeing the device specification, it is fairly self explanatory. Note that, on line~\ref{fig:os_spec_asl:l:class} the class event defined in Figure~\ref{fig:dev_spec_asl} is used.

As with the device specification, the OS specification may be seen as a Mealy machine as shown in Figure~\ref{fig:os_spec_mealy}. Again, there is a single register for storing the current state and this state is updated on each transition using combinational logic. In addition to taking the label variables as input, it also takes the \code{classSent} boolean signal that was produced by the device Mealy machine as input. 

\begin{figure}
\centering
\begin{subfigure}[t]{0.47\textwidth}
\includegraphics[width=\linewidth]{diagrams/exampleSpecOS.pdf}
\caption{OS specification}
\label{fig:os_spec}
\end{subfigure}
\hfill
\begin{subfigure}[t]{0.47\textwidth}
\centering
\includegraphics[width=\linewidth]{diagrams/osMealy.pdf}
\caption{OS specification}
\label{fig:os_spec_mealy}
\end{subfigure}
\par\bigskip
\begin{subfigure}[b]{\textwidth}
\centering
\begin{asllisting}
State
osState  : {osIdle, osRequested, osDone, osError};

Label
osReqSend    : bool;
controllable : bool;

Init
osState == osIdle 

Transitions
osState := case {
    osState == osIdle :
        case {
            classSent                 : osError;
            osReqSend && controllable : osRequested;
            true                      : osState;
        };
    osState == osRequested :
        case {
            classSent : osDone; (*@\label{fig:os_spec_asl:l:class}@*)
            true      : osState;
        };
    osState == osDone  : osDone;
    osState == osError : osError;
};
\end{asllisting}
\caption{Symbolic OS specification}
\label{fig:os_spec_asl}
\end{subfigure}
\end{figure}

\subsection{Combined Specification}

We combine the device and OS specifications by executing them concurrently. We do this by instantiating both Mealy machines in parallel and connecting the signals, as shown in Figure~\ref{fig:combined_spec_mealy}. In ASL, this is as simple as placing both case statements in the \code{Transitions} section, combining the state and label declaration sections and combining the initial states. The result is given in Figure~\ref{fig:combined_spec_asl}.

\begin{figure}
\centering
\begin{asllisting}
State
osState  : {osIdle, osRequested, osDone, osError};
devState : {devIdle, devSending};

Label
osReqSend    : bool;
classSent    : bool;
devSendReq   : bool;
controllable : bool;

Init
osState == osIdle && devState == devIdle

Transitions
devState := case {
    devState == devIdle && devSendReq && controllable  : devSending;
    devState == devSending && devSent && !controllable : devIdle;
    true                                               : devState;
}

define(classSent, (devState == devSending && devSent && !controllable)) 

osState := case {
    osState == osIdle :
        case {
            classSent                 : osError;
            osReqSend && controllable : osRequested;
            true                      : osState;
        };
    osState == osRequested :
        case {
            classSent : osDone; 
            true      : osState;
        };
    osState == osDone  : osDone;
    osState == osError : osError;
};
 
\end{asllisting}
\caption{Combined ASL specification}
\label{fig:combined_spec_asl}
\end{figure}

Figure~\ref{fig:combined_spec} shows the state machine of the combined specification. This large and cumbersome state machine clearly shows the size and complexity advantages of using separate Mealy machines in parallel as the specification language.

\begin{figure}
\centering
\includegraphics[width=0.85\linewidth]{diagrams/exampleSpecCombined.pdf}
\caption{Combined specification}
\label{fig:combined_spec}
\end{figure}

\begin{figure}
\centering
\includegraphics[width=0.85\linewidth]{diagrams/combinedMealy.pdf}
\caption{Combined specification}
\label{fig:combined_spec_mealy}
\end{figure}

\section{Concurrent Games}
\label{sec:conc_games}

You may have noticed that in Figure~\ref{fig:combined_spec} our game is not turn based. There are states in which both player~1 and player~2 have moves. This kind of game is called a \emph{concurrent game}.

Concurrent games differ from simpler turn based games in that in each state both players get to pick a label and the next state that the game transitions to is some (possibly non-deterministic) function, $\lambda$, of both of those labels. Turn based games are a special case of concurrent games where in player~i states (a concept that does not generally exist in concurrent games) the next state is entirely determined by the label played by player~i and the other player's label is ignored. 

In Termite, however, the concurrent game is almost as simple as a turn based game. Labels, not states, are classified as controllable or uncontrollable. $\lambda$ chooses the effective label non-deterministically. This means that while player~1 may choose a label in any state, there is no guarantee that it will be played. There is, however, a fairness guarantee that each player eventually gets a turn.

We simulate this non-determinism with the input variable \code{controllable}. This variable is always chosen non-deterministically. Our (now deterministic) $\lambda$ function picks the player~1 chosen label if \code{controllable} is $True$ and it picks the player~2 chosen label otherwise.

\subsection{Termite's controllable predecessor}

In this section we develop the controllable predecessor for Termite's concurrent game. Disregarding fairness for now, given a target state $X$, we define a state $s$ to be winning if both:
\begin{enumerate}
    \item there exists a controllable label originating from $S$ such that all transitions with this label lead to $X$, and
    \item all uncontrollable transitions with any label originating from $S$ lead to $X$
\end{enumerate}
\noindent are simultaneously true. 

Condition~1 is captured by the function $CpreC$ given in Equation~\ref{eqn:termite_cpre_c}. It returns the set of states from which there exists a controllable label such that all transitions with this label lead to a state in the target set $X$.

\begin{equation}
    CpreC(X) =  \exists L. controllable \land \forall N. Trans(S, L, N) \rightarrow X(N)
    \label{eqn:termite_cpre_c}
\end{equation}

Condition~2 is captured by the function $CpreU$ given in Equation~\ref{eqn:termite_cpre_u}. It returns the set of states from which all transitions with uncontrollable labels lead to a state in $X$.

\begin{equation}
    CpreU(X) =  \forall L. \neg controllable \rightarrow \forall N. Trans(S, L, N) \rightarrow X(N)
    \label{eqn:termite_cpre_u}
\end{equation}

The final controllable predecessor, $Cpre$, given in Equation~\ref{eqn:termite_cpre} returns the set of states for which both conditions are satisfied.

\begin{equation}
    Cpre(X) =  CpreC(X) \land CpreU(X)
    \label{eqn:termite_cpre}
\end{equation}

\section{GR(1) based formalism}

In the following sections, we attempt to formalise the driver synthesis problem using the simplest game - the reachability game - and analyse its shortcomings. We show that extending our formalism to GR(1) objectives, as used in Termite, overcomes these shortcomings.

\subsection{Reachability}

As a concrete example, we could create a crude formalism for driver synthesis using only a reachability game. Consider, for example, figure~\ref{fig:reach}, which shows the state machine for a game to control a hypothetical network controller. Solid lines indicate controllable transitions and dashed lines indicate uncontrollable transitions. Execution begins in the leftmost state where the OS may initiate a network transfer by choosing the `send' label. The goal of the game is the rightmost state (labelled `G') as this is the point where player 1 has completed the request. So, to win, player 1 (who controls the transitions with solid lines) must ensure that execution of the state machine reaches the goal. 

\begin{figure}
\centering
\includegraphics[width=0.85\linewidth]{diagrams/reachGame.pdf}
\caption{Reachability game for simple network device}
\label{fig:reach}
\end{figure}

The network device has two 8-bit registers, command (abbreviated cmd) and data. Writing 0x01 to the command register starts the transfer, and eventually whatever is in the data register gets written out to the network. Note that the actual sending of the data is an uncontrollable event. 

The correct sequence to win the game, therefore, is to write the data register and then the control register after the OS performs a send request. This takes us to state `S5' where the only move by player 2 is `evt\_send' taking us to the goal. 

If the command register is written first and then the data register there is potential for the environment to play the `evt\_send' label before the data is written, potentially resulting in the wrong data being sent. This is the transition that terminates in the `E' state (for error). The `E' state is a dead end, so it is not possible to reach the goal. 

So, if player 1 takes the top half of the diamond (i.e.\ writes data before command) then it will be guaranteed to reach the goal and the reachability game is winning for player 1. The strategy to reach the goal tells us the sequence of labels the driver must play to get to the goal. In principle, this could be turned into a driver for our simple network device.

This simplistic formalism for driver synthesis has several shortcomings that we will deal with in the following sections.

\subsection{\buchi}

Consider a simplified network controller that does not have a command register. Instead, writing to the data register triggers transmission of the byte. However, there are two ways of writing to the data register. One is a standard register write. The other also performs the register write and then schedules a self destruct sequence to happen immediately after the byte is transmitted. The state machine for this device is shown in figure~\ref{fig:buchi}. The goal, in this case, is the set ${S3, S5}$ corresponding to the state after completion of the send request. The problem is that, unless you only ever want to send one byte, this goal does not capture the required behavior. One could easily work around this problem by specifying only ${S3}$ as the goal, but this breaks the compositionality of the specifications.

The solution is to modify the objective of the game. Instead of being able to reach the goal once, we want to be able to reach the goal an infinite number of times. Or, equivalently, we want to always be able to reach the goal again. This kind of objective is called a Buchi objective and a game with a Buchi objective is called a Buchi game. 

\begin{figure}
\centering
\includegraphics[width=0.85\linewidth]{diagrams/buchiGame.pdf}
\caption{Buchi game for simple network device}
\label{fig:buchi}
\end{figure}

\subsection{Fairness}

Consider a modification of our simplified network device without a self destruct sequence, but with the ability to check that noone is using the communication medium prior to transmitting. The state machine of this device is given in figure~\ref{fig:fair}. After the user requests data transmission by writing to the data register, it executes a loop that checks if the medium is free, and if so, it performs the transmission. 

If we pose this as a reachability game with goal state $G$, then the game is not winnable. The device may stay in the loop forever as it is never guaranteed to exit. Such a behavior should not prevent a driver from being synthesized providing that we have good reason to believe that the loop will eventually exit. Looping forever can be seen as a invalid behavior and we want to synthesize a driver for this system providing the invalid behavior does not occur. 

In model checking these behaviors are eliminated with fairness conditions. Fairness conditions are sets of states which we guarantee will eventually be left, which we refer to as unfair states. In the example, the unfair states are the set ${S2, S3}$. The fairness condition says that we will eventually leave the unfair set, and the only way of doing this is through the $evt\_send$ transition, and the game becomes winning.

\begin{figure}[t]
\centering
\includegraphics[width=0.85\linewidth]{diagrams/fairReach.pdf}
\caption{Fair reachability game for simple network device}
\label{fig:fair}
\end{figure}

\subsection{Multiple Goals}

\subsection{Multiple Fairness}

\subsection{GR(1)}

The combination of fairness and buchi objectives is called a GR(1) objective. Intuitively a GR(1) objective says that we can always reach some goal state provided that we do not get stuck forever in some unfair set of states. We use GR(1) objectives in Termite as we have found that in practice it is sufficient to express our goals.

