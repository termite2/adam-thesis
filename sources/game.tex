\chapter{Driver Synthesis as a Game}

In this chapter we formalise the device driver synthesis problem using games. We show that GR(1) games are sufficient to capture the properties of the drivers that we require. We also develop the driver synthesis controllable predecessor.

\section{The Players}

To formalise the driver synthesis problem using games, the first thing we must define is the players:

\begin{itemize}
    \item Player~1 is the driver. 
    \item Player~2 is the environment, which consists of the device to be controlled as well as the operating system.
\end{itemize}

\todo{A pretty picture}

\section{Concurrent Games}
\label{sec:conc_games}

Our game formalism for termite makes use of concurrent games. Concurrent games differ from simpler turn based games in that in each state both players get to pick a label and the next state that the game transitions to is some (possibly non-deterministic) function of both of those labels. Turn based games are a special case of concurrent games where in player~i states the next state is entirely determined by the label played by player~i and the other player's label is ignored. 

It is important to specify which player gets to pick their label first and if the second player gets to have knowledge of the label that the first player picks when choosing their label. In Termite, player~1 (the driver) has to pick first and the environment (device and operating system) gets to pick second with knowledge of the label that the driver picked. This makes the game more difficult to win for the driver.

We call the part of the label that the driver gets to pick $C$ (for controllable) and the part that the environment gets to pick $U$ (for uncontrollable). Our transition relation is defined over the current state $S$, as well as the label $C$ and $U$. Our controllable predecessor becomes: \todo{bullshit}

\begin{equation}
    CPre(X) = \exists C. \forall U. \forall N. TRANS(S, C, U) \rightarrow X'
\end{equation}

We define a special variable that is part of the environment's label called \textsc{Turn}.

\section{A specification Language for Symbolic Games}

We describe a simple specification language for symbolic games. This language is used to illustrate, in the simplest possible way, how Termite's specifications work. Furthermore, I have created a compiler for this language as an alternate frontend for Termite and it is available as part of Termite. As a result, all of the examples given in this chapter will compile and synthesize with Termite.

Fundamentally, symbolic transition systems are specified by defining how each state variable is updated on each transition, as a function of the other state variables and the label variables. We call these functions update functions and there is one update function for each state variable. 

Our simple language consists of four sections for defining a transition system:
\begin{itemize}
    \item The state variable declaration
    \item The label variable declaration
    \item The set of initial states
    \item Variable update functions, one for each state variable
\end{itemize}

Variables in both the state and label sections may be declared as one of three types:
\begin{itemize}
    \item Fixed width integer
    \item Boolean
    \item Enumeration
\end{itemize}

The transition relation is specified using nested case statements.

The BNF grammar is given below.

\subsection{Identifiers}

An identifier is a name of a variable or a constant declared as part of an enumeration.

\begin{bnflisting}
    <ident>    := <letter> (<letter> | <digit> | "_")*
\end{bnflisting}

\subsection{Numbers}

Numbers may be specified in hexadecimal, octal or decimal.

\begin{bnflisting}
    <number>    := "0x" <hexNumber>
                 | "0o" <octNumber>
                 | <decNumber>
\end{bnflisting}

\subsection{Types}

Our specification language supports three different types: Booleans, fixed width bit vectors, and enumerations.

\begin{bnflisting}
    <boolTyp>  := "bool"
    <intTyp>   := "uint" "<" <digit>* ">"
    <enumTyp>  := "{" identifier* "}"
\end{bnflisting}

\subsection{Variables}

Variables are declared by giving a name of the variable and its type.

\begin{bnflisting}
    <type>     := <boolTyp> | <intTyp> | <enumTyp>
    <varDecl>  := <identifier> ":" <type>
\end{bnflisting}

\subsection{Boolean Expressions}

Boolean expressions consist of boolean literals, value expressions (to follow) combined with the equality operator, and other boolean expressions combined with logical connectives.

\begin{bnflisting}
    <boolLit>  := "true" | "false"
    <binExpr>  := <boolLit> 
                | <valExpr> "==" <valExpr> 
                | <binExpr> "||" <binExpr> 
                | <binExpr> "&&" <binExpr> 
                | "!" <binExpr>
\end{bnflisting}

\subsection{Value Expressions}
Each value expression is a number, an identifier or a case statement. A case statement consists of a list of cases. Each case consists of a boolean condition followed by a value expression. Thus, case statements may contain further case statements. The outcome of a case statement is the value of the first case whose boolean expression evaluates to true.

\begin{bnflisting}
    <caseStmt> := "case" "{" 
                      (<binExpr> ":" <valExpr> ";")*
                  "}"
    <valExpr>  := <caseStmt> | <number> | <identifier>
\end{bnflisting}

\subsection{Variable updates}

Update functions are specified by assigning the next state value of a state variable to a value expression.

\begin{bnflisting}
    <update>   := <identifier> ":=" <valExpr>
\end{bnflisting}

\subsection{Transition System}

Transition systems are defined by giving the set of state variables, the set of label variables, the initial states and the variable update functions. The set of initial states is specified by the characteristic formula of the set.

\begin{bnflisting}
    <spec>     := "State"       <varDecl>* 
                  "Label"       <varDecl>* 
                  "Init"        <binExpr>
                  "Transitions" <update>*
\end{bnflisting}

\section{Device and OS specifications and synchronization}

A core concept of Termite is the separation of device and operating system specifications. Games so far have been defined to have a single state machine that contains all of the states of the system as well as all of the actions. In Termite, we decouple the description of the operating system from the device to be controlled and then combine these to form a state machine before synthesis.

Both the device and OS specifications are given as state machines. They are defined symbolically. Each specification has its own set of state variables that it must supply update functions for, however, one specification may refer to another's state variables in its state update functions. There is also a global set of label variables which each specification may also refer to in its state update functions. An overview of a system consisting of a device and OS specification as well as their interconnections is given in figure~\ref{fig:specs}.

As the figure shows, device and OS specifications are identical to modules in a hardware description language such as Verilog or VHDL\@. They consist of some private state, represented by D type flip-flops (though the variables are not restricted to single bits) that is updated solely by logic within that module. Other modules can read, but not write to, this private state. The label, that is chosen by one of the players, appears as input to both of the modules and they can use it in the update functions for their private state.

To ensure reusability of specifications, specifications do not directly refer to each other's state variables. Instead, they access them through well defined interfaces and there is a device class mechanism to standardise the interfaces of both the device and operating system specification interfaces. We delay description of this mechanism until Chapter~\ref{c:userguided}. For now we assume that specifications can access each other's state variables without restriction.

\section{An Example System Specification}

We give an example system specification to illustrate its decomposition into separate OS and device specifications. We give the system both as graphical state machines and in ASL.

\subsection{State Machines}

Figure~\ref{fig:dev_spec} is our example device specification. The labelled circles are states. Arrows represent transitions between states. Solid arrows are controllable transitions whereas dashed arrows are uncontrollable, as described in Section~\ref{sec:conc_games}. Each transition is labelled by at least one event. Conceptually, these events trigger the transitions. All events for which there are no outgoing transitions do not change the state, i.e.\ they are self loops and are not shown in the interest of keeping the diagrams concise. The short arrow with no source state indicates the initial state. There may be more than one initial state, in which case the initial state is chosen non-deterministically.

\subsection{Device Specification}

We start with the device specification given in Figure~\ref{fig:dev_spec}. Our device is a hypothetical trivial UART-like device. Instead of sending bytes, it sends notifications, which carry no value. Its programming interface consists of an action, \code{devSendReq}, that requests that the device send a notification. In a real system this could be a write to a particular bit in a control register that triggers the device action. This action only has an effect in the \code{devIdle} state and it causes the device to transition to the \code{devSending} state. From there, when the \code{devSent} event happens, which corresponds to the device actually sending the notification, the device transitions back to the \code{devIdle} state. Additionally, it emits the \code{classSent} event. We will respond to the \code{classSent} event in the OS specification in the next section. From here, another notification may be requested by performing the \code{devSendReq} action again.

A fragment of the symbolic encoding of the device specification is given in Figure~\ref{fig:dev_spec_asl}. There is only one state variable, \code{devState}, that toggles between two values: \code{devIdle} and \code{devSending}. Additionally, there are three boolean label variabes: \code{controllable}, \code{devSendReq} and \code{devSent}. The \code{devSendReq} and \code{devSent} variables correspond to the actions in the provious paragraph. The \code{controllable} variable is an additional variable to distinguish between controllable and uncontrollable events. 

The update function for the \code{devState} variable (the only variable in the device specification) consists of a single case statement. If the system is currently in the \code{devIdle} state and the \code{devSendReq} label variable is true, and, importantly, the \code{controllable} label variable is also true, then the system transitions to the \code{devSending} state. The \code{controllable} variable had to be true because this was a controllable transition. Additionally if the system is in the \code{devSending} state and the \code{devSent} variable is true and the transition is not controllable then the system transitions to the \code{devIdle} state. If neither of these conditions are met, the device remains in the same state. 

On line~\ref{fig:dev_spec:l:output} we defined the \code{classSent} event to occur when the device is in state \code{devSending}, the \code{devSent} label is true and the transition is not controllable. This matches the state machine. The \code{define} statement is actually an m4~\cite{m4} macro that will be used to substitute the definition of \code{classSent} into the operating system specification later. This behaves much like a signal or continuous assignment in a hardware description language, but, to keep our specification language minimal we have not implemented signals.

Lastly, this device model may be seen as a hardware block as shown in Figure~\ref{fig:dev_spec_block}. There is a single register for storing the current state and this state is updated on each transition using combinational logic which additionally takes the label variables as input. This combinational logic also produces the output boolean variable \code{classSent}.

\begin{figure}
\centering
\includegraphics[width=0.45\linewidth]{diagrams/exampleSpecDevice.pdf}
\caption{Device specification}
\label{fig:dev_spec}
\end{figure}

\begin{figure}
\centering
\begin{asllisting}
State
devState : {devIdle, devSending};

Label
devSendReq   : bool;
devSent      : bool;
controllable : bool;

Init
devState == devIdle

Transitions
devState := case {
    devState == devIdle && devSendReq && controllable  : devSending;
    devState == devSending && devSent && !controllable : devIdle;
    true                                               : devState;
}

define(classSent, (devState == devSending && devSent && !controllable)) (*@\label{fig:dev_spec:l:output}@*)

\end{asllisting}
\caption{Symbolic device specification}
\label{fig:dev_spec_asl}
\end{figure}

\subsection{OS specification}

The operating system specification, given in Figure~\ref{fig:os_spec} is like a test harness for our driver. It specifies which requests the driver may receive (e.g.\ to send a notification) and how it must respond (e.g.\ by eventually sending the requested notification). It is simply another state machine, like the device specification, but it may also have goals.

The goal state, state \code{O3}, is the state that a correct driver will eventually force the OS specification to be in. It is possible to specify multiple goal states, in which case the driver must force execution into any goal state.

Our OS state machine starts off in the \code{osIdle} state. From there, if a \code{classSent} event is triggered by the device state machine, then the device must have sent a notification without the OS ever having requested one. This is an error, so the OS state machine transitions to the \code{osError} state. There is no way that the OS could have known that this event happened as it does not monitor the device notification output, so, the OS specification does not represent the way that the OS behaves in reality. These events that do not correspond to real interactions but are used to enforce correctness are called \emph{virtual events}. Responding to virtual events in this way is how the OS spec guarantees correctness of the system. 

There is nothing special about the error state, however, we have constructed the state machine in such a way that the error state is a dead end and it is not possible to reach the goal from that state.

If a \code{devSndReq} event happens, the OS transitions into the \code{devRequested} state. From there, when a \code{classSent} event happens, the OS specification transitions into the goal state as this time the \code{classSent} event was expected. 

We can see from the OS specification that it is the job of the driver to ensure that a \code{devSendReq} event is eventually followed by \code{classSent} event. However, a \code{classSent} event can only be issued by the device, not the driver. The driver can, however, force the device to issue a \code{classSent} event indirectly in response to a \code{devSendReq} event. 

In general, it is the job of the driver synthesis algorithm to figure out how to force certain events to happen at the right time (as specified by the OS state machine) by using information from the device state machine. 

\begin{figure}
\centering
\includegraphics[width=0.85\linewidth]{diagrams/exampleSpecOS.pdf}
\caption{OS specification}
\label{fig:os_spec}
\end{figure}

\begin{figure}
\centering
\begin{asllisting}

State
osState  : {osIdle, osRequested, osDone, osError};

Label
osReqSend    : bool;
controllable : bool;

Init
osState == osIdle 

Transitions
osState := case {
    osState == osIdle :
        case {
            classSent                 : osError;
            osReqSend && controllable : osRequested;
            true                      : osState;
        };
    osState == osRequested :
        case {
            classSent : osDone;
            true      : osState;
        };
    osState == osDone  : osDone;
    osState == osError : osError;
};

\end{asllisting}
\caption{Symbolic OS specification}
\label{fig:os_spec_asl}
\end{figure}

\subsection{Combined Specification}

To combine the device and OS specifications we need to be precise about what it means to synchronize on events. We modify the specifications to remove all events and replace them with inputs and outputs.

\begin{figure}
\centering
\includegraphics[width=0.85\linewidth]{diagrams/exampleSpecCombined.pdf}
\caption{Combined specification}
\label{fig:combined_spec}
\end{figure}

\subsection{ASL}

Below, we give the specification of a simple UART-like device that capable of sending a notification (which does not contain a value as a real UART would) through a communications channel. Both the OS and device specifications are combined. \todo{Should I show them separated first?}

\begin{asllisting}
State

osState  : {osIdle, osRequested, osDone, osError};
devState : {devIdle, devSending};

Label

osReqSend  : bool;
classSent  : bool;
devSendReq : bool;

Init

osState == osIdle && devState == devIdle

Transitions

devState := case {
    devState == devIdle    && devSendReq : devSending;
    devState == devSending && classSent  : devIdle;
    true                                 : devState;
}

define(sentEvent, (devState == devSending && classSent))
 
osState := case {
    osState == osIdle :
        case {
            classSent : osError;
            osReqSend : osRequested;
            true      : osState;
        };
    osState == osRequested :
        case {
            classSent : osDone;
            true      : osState;
        };
    osState == osDone  : osDone;
    osState == osError : osError;
};
\end{asllisting}

This specification deserves some explanation. Initially, we declare two state variables \code{osState} and \code{devState} for the OS controlled and device controlled state respectively. Each is declared as an \code{enum} with several possible values. In the \code{Init} section they are initialised to their designated initial values.

Additionally, we declare three label variables: \code{osReqSend}, \code{classSent} and \code{devSentReq} that are treated as input and are selected by the players.

There are two update functions in the \code{Transitions} section; one for each state variable. The first, for \code{devState}, toggles the enumeration between one of two states: \code{devIdle} and \code{devSending}. Initially \code{devIdle}, the variable is updated to \code{devSending} when a player sets the label \code{devSendReq} to True. It is reset when a player sets \code{classSent} to true. In all other cases, its value remains the same as in the previous state.

We additionally define the \code{sentEvent} event which evaluates to true whenever the device is in the \code{devSending} state and a player sets the \code{classSent} label to true.

The state machine for the device is given in Figure x. \todo{Would it be better to give the device state machine with class event, then modify that so that the class event is an input, and then the specification?}

\section{GR(1) based formalism}

In the following sections, we attempt to formalise the driver synthesis problem using the simplest game - the reachability game - and analyse its shortcomings. We show that extending our formalism to GR(1) objectives, as used in Termite, overcomes these shortcomings.

\subsection{Reachability}

As a concrete example, we could create a crude formalism for driver synthesis using only a reachability game. Consider, for example, figure~\ref{fig:reach}, which shows the state machine for a game to control a hypothetical network controller. Solid lines indicate controllable transitions and dashed lines indicate uncontrollable transitions. Execution begins in the leftmost state where the OS may initiate a network transfer by choosing the `send' label. The goal of the game is the rightmost state (labelled `G') as this is the point where player 1 has completed the request. So, to win, player 1 (who controls the transitions with solid lines) must ensure that execution of the state machine reaches the goal. 

\begin{figure}
\centering
\includegraphics[width=0.85\linewidth]{diagrams/reachGame.pdf}
\caption{Reachability game for simple network device}
\label{fig:reach}
\end{figure}

The network device has two 8-bit registers, command (abbreviated cmd) and data. Writing 0x01 to the command register starts the transfer, and eventually whatever is in the data register gets written out to the network. Note that the actual sending of the data is an uncontrollable event. 

The correct sequence to win the game, therefore, is to write the data register and then the control register after the OS performs a send request. This takes us to state `S5' where the only move by player 2 is `evt\_send' taking us to the goal. 

If the command register is written first and then the data register there is potential for the environment to play the `evt\_send' label before the data is written, potentially resulting in the wrong data being sent. This is the transition that terminates in the `E' state (for error). The `E' state is a dead end, so it is not possible to reach the goal. 

So, if player 1 takes the top half of the diamond (i.e.\ writes data before command) then it will be guaranteed to reach the goal and the reachability game is winning for player 1. The strategy to reach the goal tells us the sequence of labels the driver must play to get to the goal. In principle, this could be turned into a driver for our simple network device.

This simplistic formalism for driver synthesis has several shortcomings that we will deal with in the following sections.

\subsection{\buchi}

Consider a simplified network controller that does not have a command register. Instead, writing to the data register triggers transmission of the byte. However, there are two ways of writing to the data register. One is a standard register write. The other also performs the register write and then schedules a self destruct sequence to happen immediately after the byte is transmitted. The state machine for this device is shown in figure~\ref{fig:buchi}. The goal, in this case, is the set ${S3, S5}$ corresponding to the state after completion of the send request. The problem is that, unless you only ever want to send one byte, this goal does not capture the required behavior. One could easily work around this problem by specifying only ${S3}$ as the goal, but this breaks the compositionality of the specifications.

The solution is to modify the objective of the game. Instead of being able to reach the goal once, we want to be able to reach the goal an infinite number of times. Or, equivalently, we want to always be able to reach the goal again. This kind of objective is called a Buchi objective and a game with a Buchi objective is called a Buchi game. 

\begin{figure}
\centering
\includegraphics[width=0.85\linewidth]{diagrams/buchiGame.pdf}
\caption{Buchi game for simple network device}
\label{fig:buchi}
\end{figure}

\subsection{Fairness}

Consider a modification of our simplified network device without a self destruct sequence, but with the ability to check that noone is using the communication medium prior to transmitting. The state machine of this device is given in figure~\ref{fig:fair}. After the user requests data transmission by writing to the data register, it executes a loop that checks if the medium is free, and if so, it performs the transmission. 

If we pose this as a reachability game with goal state $G$, then the game is not winnable. The device may stay in the loop forever as it is never guaranteed to exit. Such a behavior should not prevent a driver from being synthesized providing that we have good reason to believe that the loop will eventually exit. Looping forever can be seen as a invalid behavior and we want to synthesize a driver for this system providing the invalid behavior does not occur. 

In model checking these behaviors are eliminated with fairness conditions. Fairness conditions are sets of states which we guarantee will eventually be left, which we refer to as unfair states. In the example, the unfair states are the set ${S2, S3}$. The fairness condition says that we will eventually leave the unfair set, and the only way of doing this is through the $evt\_send$ transition, and the game becomes winning.

\begin{figure}[t]
\centering
\includegraphics[width=0.85\linewidth]{diagrams/fairReach.pdf}
\caption{Fair reachability game for simple network device}
\label{fig:fair}
\end{figure}

\subsection{Multiple Goals}

\subsection{Multiple Fairness}

\subsection{GR(1)}

The combination of fairness and buchi objectives is called a GR(1) objective. Intuitively a GR(1) objective says that we can always reach some goal state provided that we do not get stuck forever in some unfair set of states. We use GR(1) objectives in Termite as we have found that in practice it is sufficient to express our goals.

