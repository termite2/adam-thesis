\documentclass{book}

\usepackage[margin=1in]{geometry}
\usepackage{algorithm}
\usepackage{algpseudocode}
\usepackage{graphicx}
\usepackage{amsmath}
\usepackage{amssymb}

\title{Device Driver Synthesis}
\author{Adam Walker}

\newcommand{\buchi}{Buchi }
\newcommand{\reach}[0]{\textsc{Reach}}
\newcommand{\safe}[0]{\textsc{Safe}}
\newcommand{\concrete}[1]{#1\mathord{\downarrow}}
\newcommand{\abstractm}[1]{#1\mathord{\uparrow^m}}
\newcommand{\abstractM}[1]{#1\mathord{\uparrow^M}}
\newcommand{\forms}[0]{\mathcal{F}}

\begin{document}

\maketitle
\tableofcontents

\chapter{Introduction}

\section{Termite}
\section{Contributions}
\section{Chapter outline}

\chapter{Background}

\section{$\mu$-calculus}

\subsection{Syntax}

The $\mu$-calculus is a logic capable of expressing greatest and least solutions of fixed point equations $X = f(X)$ where $f$ is a monotone function. The set of formulas of $\mu$-calculus is defined as follows:

Let $P$ be a set of propositions, $A$ be a set of actions and $V$ be a set of variables. Then
\begin{itemize}
    \item each proposition and each variable is a formula
    \item if $\phi$ and $\psi$ are formulas then $\phi \wedge \psi$ is a formula
    \item if $\phi$ is a formula then $\neg \phi$ is a formula
    \item if $\phi$ is a formula and $a$ is and action then $[a]\phi$ is a formula
    \item if $\phi$ is a formula and $Z$ a variable then $\nu X.\phi$ is a formula provided that every free occurrence of $Z$ occurs under an even number of negations
\end{itemize}

\subsection{Semantics}

Given a labelled transition system $(S, R, V)$ where
\begin{itemize}
    \item S is a set of states
    \item $R \subseteq S \times A \times S$
    \item $V : Var \rightarrow 2^S$ maps each proposition to the set of states where the proposition is true
\end{itemize}

A $\mu$-calculus formula is interpreted as follows:
\begin{itemize}
    \item $p$ holds in the set of states $V(p)$
    \item $\phi \wedge \psi$ holds in the set of states where both $\phi$ and $\psi$ hold
    \item $\neg \phi$ holds in every state where $\phi$ does not hold 
    \item $[a]\phi$ holds in a state $s$ if every $a$-transition leading out of $s$ leads to a state where $\phi$ holds
    \item $\nu Z. \phi$ holds in any set of states $T$ such that when the variable $Z$ is set to $T$ then $\phi$ holds for all of $T$. It is the greatest fixpoint of $\phi$.
\end{itemize}

\section{Two player games}

Two-player games are a useful formalism for reactive synthesis. Many problems in electronic design automation, industrial automation and robotics can be formalised as a game. In particular, the driver synthesis problem can be formalised as a game, and, this is the formalism around which Termite is built. Here, we present the fundamentals of $\omega$ regular games. 

\subsection{Formalism}

A two player game is played by player~1 against its opponent, player~2. It consists of a possible infinite state space $S$ on which the game is played. The game is always in some state $s \in S$ called the current state. The game progresses from state to state according to a transition relation, $\delta \subseteq S \times L \times S$ where $S$ is the set of states and $L$ is a set of label variables. A transition $t \in S \times L \times S$ is allowed in the game iff $t \in delta$. 

The meaning of the label $l \in L$ depends on the type of game, but for now we will consider turn based games. In a turn based game, $S$ is partitioned into two sets: the player~1 set $\tau_1$ and the player~2 set $\tau_2$, where $\tau_1 \cap \tau_2 = \emptyset$ and $ \tau_1 \cup \tau_2 = S$. When $s \in \tau_1$ player 1 gets to pick $l$ and when $s \in \tau_2$ player 2 gets to pick $l$.

Lastly, each game has an associated set of initial states $I \in 2^S$ where execution of the game begins.

Putting this all together, we can identify a \emph{turn based game structure} $G = \langle S,L,I,\tau_1,\tau_2,\delta \rangle$ with a turn based game.

A game proceeds in an infinite sequence of rounds, starting from some initial state. The infinite sequence of states visited $(s_0, s_1,\ldots) \in S^\omega$ is called a path. 

A game is won by player 1 if, by picking suitable labels, they can force the path to be within a winning set of state objectives $\Phi \subseteq S^\omega$, where $S^\Omega$ is the set of all infinite sequences of states in $S$. An arbitrary set of infinite sequences is an extremely general, but not practically useful, way of defining an objective. In the following sections we will consider some more restricted objectives that have practical uses.

\subsection{Safety and reachability games}

The two simplest objectives are safety and reachability. A safety objective is defined by a set $SAFE \subseteq S$ that player 1 must force the game to stay within, regardless of the labels that player 2 picks. Formally, a run is safe if $\forall i. s_i \in SAFE$. 

The dual of a safety objective is a reachability objective. A reachability objective is defined by a set $REACH \subseteq S$ that player 1 must force the game to visit at least once, regardless of the labels that player 2 picks. Formally, a reachability run is winning if $\exists i. s_i \in REACH$

\subsection{Strategies, counterexamples}

Given an objective, a game is either winning for player~1 or it is losing. If it is winning, then there exists a strategy for player~1. Informally, a strategy tells player~1, in any state, which label it must play and ensures that if player~1 adheres to the strategy then the objective will be satisfied.

Formally, a strategy for player~1 is a function $\pi_1 : S^* \times \tau_1 \rightarrow L$ that, in any player~1 state, associates the history of the game with a label to play. It is known that for complete information games, as we are considering, if there exists a strategy, then there also exists one that only depends on the current state. Thus, we can simplify the definition of a strategy for player~1 to be a function $\pi_1 : \tau_1 \rightarrow L$ that only depends on the current state.

Conversely, if the game is losing for player~1, then there exists a spoiling strategy for player~2, that is, a strategy for player~2 that ensures that, no matter which labels player~1 picks, there is no way that player~1 can satisfy its objective.

Formally, a counterexample strategy for player~1 is a strategy for player~2 which is a function $\pi_2 : \tau_2 \rightarrow L$ that associates any player~2 state with a label for player~2 to play. Again, we have simplified the definition as there must exist a strategy that only depends on the current state. 

\subsection{Buchi, fairness and GR(1) games}
Here we consider objectives that are more complex and more useful in practice.

\paragraph{Buchi}
A Buchi objective is defined by a set $BUCHI \subseteq S$ that player~1 must always be able to force execution of the game into. This differs from a reachability game in that the region must always be reachable, not just once. When it has been reached once, it must be reachable again, and so on. So, it must be reachable infinitely many times. Formally, a run is buchi winning if $\forall i. \exists j>i. s_j \in BUCHI$

\paragraph{Generalized Buchi}
A Generalized Buchi objective is defined by a set of sets $BUCHIS \subseteq 2^S$. Player~1 must always be able to force execution into each set $BUCHI \in BUCHIS$. Formally, a run satisfies a generalized Buchi objective if $\forall i. \forall BUCHI \in BUCHIS. \exists j>i. s_j \in BUCHI$.

\paragraph{Fairness}
Sometimes it is necessary to rule out invalid plays that are not easily ruled out by changing the state machine. For this we use fairness conditions. A fairness condition is a set of states which we may assume that all valid runs of the game eventually leave. If a spoiling strategy exists that results in an unfair run, it does not count. Formally, a run satisfies a fair reachability objective if $\forall i. \exists j>i. s_j \notin UNFAIR \rightarrow \exists i. s_i \in REACH$. 

\paragraph{GR(1)}
A Generalized Reactive 1, or GR(1) objective, is a generalized Buchi objective with multiple fairness conditions. In practice, this turns out to be a very useful type of objective. Formally, a run satisfies a GR(1) objective if $\forall i. \forall UNFAIR \in UNFAIRS \exists j>i. s_j \notin UNFAIR \rightarrow \forall i. \forall BUCHI \in BUCHIS. \exists j>i. s_j \in BUCHI$.

\subsection{Solving games}
\subsubsection{Controllable predecessor}

All of the synthesis algorithms for games I will be describing use a function called the controllable predecessor (abbreviated CPre). CPre is a function from a set of game states ($2^S$) to another set of game states. Given a set $T$, $CPre(T)$ returns the set of states from which player 1 can force execution into $T$ in one step. 

The exact details of $CPre$ do not matter and it can be considered to be a parameter to the game solving algorithms. The function it computes depends on the application and I will be describing the driver synthesis $CPre$ later. For now, the only property of $CPre$ that we require is that it is monotonic, ie. if $X \subseteq Y$ then $CPre(X) \subseteq CPre(Y)$. Clearly, any reasonable $CPre$ function will have this property.

\subsubsection{Safety and reachability}

We will start with reachability as it is the easiest to understand. A state is winning in a reachability game if there is some finite $i$ such that we can guarantee that after $i$ rounds of the game, execution will have, at least once, entered a state in $REACH$.

We can define the winning region inductively. It is obviously possible to reach the set $REACH$ from $REACH$ in 0 steps as we are already there. This is the base case. It is also possible to reach $REACH$ from $x \in S$ in $N + 1$ steps or fewer iff it is possible to reach $REACH$ from all of $Y \subseteq 2^S$ in $N$ steps or fewer and $x \in CPRE(Y)$.

This suggests an algorithm. We start with $REACH$, apply $CPRE$ to get the states winning in 1 step, combine with $REACH$ again to get the states reachable in 1 step or less, and then repeat. On each iteration we find the states winning in $N$ or less steps. Two questions remain: 

\begin{itemize}
    \item An algorithm must terminate, so, when should we stop iterating?
    \item After termination, has the algorithm found all the winning states?
\end{itemize}

Observe that $W_{N+1} \supseteq W_N$. So, on each iteration, we either grow the winning set or it remains the same. Also observe that our set $S$ is finite, which means that $2^S$ is also finite. As set inclusion is a transitive relation, we cannot grow our winning set forever, so eventually we must reach a fixed point of the CPRE function. 

%TODO: $CPRE$ isnt the order preserving function
Or, if you prefer, the power set of $S$ can be ordered by set inclusion to obtain a complete lattice with supremum $S$ and infimum $\emptyset$. $CPRE$ is an order preserving function, so by the Knaster-Tarski theorem the set of fixed points of $CPRE$ is also a complete lattice. Thus there exists a greatest and least fixed point (as the lattice is complete). The least fixed point of $CPRE$ is clearly obtained by iteration starting from the least element of the lattice, ie. $\emptyset$.

Thus, we will eventually reach a fixed point, and, this is when we should stop iterating as further iterations will not change the winning set. Furthermore, we know that after any finite number $N$ of iterations where $N$ is greater than the number of iterations required to reach the fixed point, the winning region will remain $WIN$. Thus $WIN_N = WIN$ and we have found all winning states.

Safety games are the dual of reachability games and are also solved by iterating the controllable predecessor. They will not be described here.

\begin{equation}
REACH(T) = \mu X. CPre(X \vee T)
\label{equ:mu_reach}
\end{equation}

\begin{equation}
SAFE(T) = \nu X. CPre(X \wedge T)
\label{equ:mu_safe}
\end{equation}

\begin{algorithm}[t]
\begin{algorithmic}
\Function{Reach}{$REACH$}
\State $Y \gets \varnothing$
\Loop
\State $Y' \gets CPre(Y \cup REACH)$
\If {$Y' = Y$} 
\State\Return $Y$\EndIf
\State $Y \gets Y'$
\EndLoop
\EndFunction
\end{algorithmic}
\caption{Solving a reachability game}
\label{a:reach}
\end{algorithm}

\begin{algorithm}[t]
\begin{algorithmic}
\Function{Safe}{$SAFE$, $CPre$}
\State $Y \gets S$
\Loop
\State $Y' \gets CPre(Y \cap SAFE)$
\If {$Y' = Y$} 
\State\Return $Y$\EndIf
\State $Y \gets Y'$
\EndLoop
\EndFunction
\end{algorithmic}
\caption{Solving a safety game}
\label{a:safe}
\end{algorithm}

\subsubsection{Buchi}

To solve a Buchi game, you first find the set from which you can reach the goal once as you do for a reachability game. Then, you use this set to find the set from which you can reach the goal twice, three times, and so on until you get to another fixed point. 

Imagine you have solved the reachability game $R = REACH(T)$ where $T$ is the Buchi target set and $R$ is the winning region. Then $V = R \wedge T$ is a subset of the goal from which you can reach the goal one more time. Computing $W = REACH(V)$ gets you the set of states from which you can reach the goal twice. Iterating this procedure will eventually lead to a fixed point as $REACH$ is a monotonic function.

This fixed point is the set of states from which you can reach the goal any number of times. Thus, it is the winning set of the Buchi game. The algorithm is given in algorithm \ref{a:buch}.

\begin{algorithm}[t]
\begin{algorithmic}
\Function{Buchi}{$T$, $CPre$}
\State $Y \gets S$
\Loop
\State $Y' \gets \Call{Reach}{Y \cap T}$
\If {$Y' = Y$} 
\State\Return $Y$\EndIf
\State $Y \gets Y'$
\EndLoop
\EndFunction
\end{algorithmic}
\caption{Solving a \buchi game}
\label{a:buchi}
\end{algorithm}

\begin{equation}
BUCHI(T) = \nu X. \mu Y. CPre(Y \vee (X \wedge T))
\label{eqn:mu_buchi}
\end{equation}

\subsubsection{Generalized Buchi}

Solving a generalized buchi game is similar to a buchi game except that you find the set from which you can reach any goal once, then any goal twice, etc. The algorithm is a small modification to the buchi algorithm and is given in algorithm \ref{alg:gen_buchi}.

\begin{algorithm}[t]
\begin{algorithmic}
\Function{Generalized\_Buchi}{$T$, $CPre$}
\State $Y \gets S$
\Loop
\State $Y' \gets S$

\For{\textbf{each} G in Goals}
\State $Y' \gets Y' \wedge \Call{Reach}{Y \cap G}$
\EndFor

\If {$Y' = Y$} 
\State\Return $Y$\EndIf
\State $Y \gets Y'$

\EndLoop
\EndFunction
\end{algorithmic}
\caption{Solving a generalized \buchi game}
\label{alg:gen_buchi}
\end{algorithm}

\begin{equation}
    GEN\_BUCHI(T) = \nu X. \prod_{G \in GOALS} \mu Y. CPre(Y \vee (X \wedge G))
\label{eqn:mu_buchi}
\end{equation}

\subsubsection{GR(1)}

Next, we need to add fairness. Consider a fair reachability game. An unfair region is a region that we can assume execution will leave, regardless of the loops it contains. We modify the controllable predecessor to create a fair controllable predecessor that takes this into account. Intuitively, the algorithm considers a block of unfair states to be winning if the only way out leads to an already winning state. To achieve this, we play a variation of a safety game where we can win if we stay entirely within the unfair set or upon exiting the unfair set we are immediately in the target set $T$. The procedure for the fair controllable predecessor is given in algorithm \ref{a:fair_cpre}. Using $Fair\_CPre$ as the $CPre$ operator in the reachability algorithm (algorithm \ref{a:reach}) yields the fair reachability algorithm. The reader is invited to check that this algorithm correctly determines that all states in figure \ref{fig:fair} are winning.

\begin{algorithm}[t]
\begin{algorithmic}
\Function{Fair\_CPre}{$CPre$, $T$}
\State $Y \gets S$
\Loop
\State $Y' \gets \Call{CPre}{Y \cap T}$
\If {$Y' = Y$} 
\State\Return $Y$\EndIf
\State $Y \gets Y'$
\EndLoop
\EndFunction
\end{algorithmic}
\caption{The fair controllable predecessor}
\label{a:fair_cpre}
\end{algorithm}

\begin{equation}
FAIR\_CPRE(U, T) = \nu X. CPre((U \wedge X) \vee T)
\label{eqn:mu_fair}
\end{equation}

Finally, if we combine the Buchi game with the fair controllable predecessor we get a GR(1) game. The procedure is given in algorithm \ref{a:gr1}.

\begin{algorithm}[t]
\begin{algorithmic}
\Function{GR(1)}{$CPre$, $T$}
\State\Return \Call{Buchi}{$T$, Fair\_Cpre(CPre)}
\EndFunction
\end{algorithmic}
\caption{GR(1) game}
\label{a:gr1}
\end{algorithm}

\begin{equation}
GR1(G, U) = \nu X. \mu Y. \nu Z. CPre((U \wedge Z) \vee (G \wedge X) \vee Y)
\label{equ:mu_gr1}
\end{equation}

\subsection{Extracting strategies}

\subsubsection{Reachability}

Once we have solved a game and determined that it is winning, we can extract a strategy. In driver synthesis, the strategy is used to generate the driver. A strategy is a relation between states and labels to play that are guaranteed to get you closer to the goal. Strategy extraction requires a straightforward modification to the game solving algorithm.

When extracting a strategy for a reachability game we need to record, for each iteration, how we got one step closer to the goal. The algorithm is given in algorithm \ref{alg:reach_strat}. 

The strategy relation is initialized to the empty set. Then, we perform an iteration of the controllable predecessor. This time we use \textsc{CPre\_Strat}, which, in addition to computing the next reachable set, also computes a relation between the newly discovered winning states and the labels that take execution from these newly discovered states one step closer to the goal.

On each iteration we combine this strategy for the newly discovered states with the strategy for the previously discovered states until we reach a fixed point as before. In the end, the strategy relates each state in the final winning region to a label that takes the game one step closer to the goal.

\begin{algorithm}[t]
\begin{algorithmic}
\Function{Reach\_Strat}{$REACH$}
\State $Y \gets \varnothing$
\State $STRAT \gets \varnothing$
\Loop
\State $(Y', STRAT') \gets CPre\_Strat(Y \cup REACH)$
\If {$Y' = Y$} 
\State\Return $(Y, STRAT)$\EndIf
\State $STRAT \gets STRAT \cup STRAT'$
\State $Y \gets Y'$
\EndLoop
\EndFunction
\end{algorithmic}
\caption{Extracting a strategy for a reachability game}
\label{alg:reach_strat}
\end{algorithm}

\subsubsection{Buchi}

Like a reachability strategy, a strategy for a buchi game must ensure that we eventually get to the goal. However, it must also ensure that once we get to the goal it is still possible to reach the goal again. 

A buchi strategy must guarantee that we can reach the intersection of the goal and the winning region, which will be non-empty if the game is winnable. The algorithm for computing the strategy in a buchi game is given in algorithm \ref{alg:buchi_strat}.

\begin{algorithm}[t]
\begin{algorithmic}
\Function{Buchi\_Strat}{$REACH$}
\State $win \gets \Call{Buchi}{REACH}$
\State \Return \Call{Reach\_Strat}{$win \cap REACH$}
\EndFunction
\end{algorithmic}
\caption{Extracting a strategy for a buchi game}
\label{alg:buchi_strat}
\end{algorithm}

\subsubsection{Generalized buchi}

Like a generalized buchi strategy must ensure that we can always get to any of the goals. A generalized buchi strategy must guarantee that we can reach the intersection of any of the goals and the winning region, which will be non-empty if the game is winnable. The algorithm for computing the strategy in a generalized buchi game is given in algorithm \ref{alg:gen_buchi_strat}. It returns one strategy for each goal that ensures you can get to the goal while remaining in the winning region. 

\begin{algorithm}[t]
\begin{algorithmic}
\Function{Gen\_Buchi\_Strat}{$REACH$}
\State $win \gets \Call{Generalized\_Buchi}{REACH}$

\For{\textbf{each} G in Goals}
\State \Return \Call{Reach\_Strat}{$win \cap G$}
\EndFor

\EndFunction
\end{algorithmic}
\caption{Extracting a strategy for a generalized buchi game}
\label{alg:gen_buchi_strat}
\end{algorithm}

\subsubsection{Fair reachability}

A strategy for a fair reachability game must ensure that we can reach the goal assuming that an unfair condition is not forever true. Conceptually, it tries to keep execution within an unfair region for which the only way out takes us a step closer to the goal. The algorithm for computing the strategy in a fair reachability game is given in algorithm \ref{alg:fair_reach_strat}.

\subsubsection{GR(1)}

\subsection{Extracting counterexamples}

\section{Symbolic games}

The algorithm described so far appears very inefficient. Consider a reachability game. We are performing a backwards breadth-first search starting from $REACH$. If we were to implement it directly as described, we would need a set abstract datatype to represent the winning set. Some of the games we have solved with Termite have upwards of $2^{80}$ states, even after abstraction. Clearly, explicitly representing the winning set will never succeed. 

Identical problems are encountered in model checking. The breakthrough that revolutionised model checking was to represent state sets implicitly as a characteristic equation over state variables.

\subsection{State variable encoding}

Symbolic games are defined over a finite set of state variables, $X$, and a finite set of label variables $Y$. We redefine $S$, the set of states, to be the set of possible valuations of each state variable in $X$. That is, each state $s \in S$ is given by a valuation of all of the state variables in $X$. Similarly, we redefine $L$, the set of labels, to be the set of possible valuations of each label variable in $Y$.

For a set $Z$ of variables, we denote by $\forms(Z)$ the set of propositional formulas constructed from the variables in $Z$. The characteristic formula of a set of states $T$ is a function $f \in \forms(X)$ that evaluates to true for the valuation corresponding to a state $s \in T$ and false otherwise. We use characteristic formulas to represent sets of states as it is more compact than explicitly listing each member of the set. This is called a symbolic representation.

Likewise, $\delta$ is specified by a formula in $\forms(X \bigcup Y \bigcup X')$, where $X' = \{x' \mid x \in X \}$ is the set of next state variables.

\subsection{Symbolic algorithms}

We can redefine Algorithms 1-4 using characteristic functions instead of explicit sets. The algorithms are superficially similar except they use conjunction and disjunction to modify sets instead of explicit set intersection and union. TODO: should I show the algorithms again with symbolicness?

\subsection{Strategy generation}
\subsection{Binary decision diagrams}

A binary decision diagram is an efficient data structure for manipulating large propositional formulas. 

\subsubsection{Binary decision trees}

\begin{figure}[t]
\centering
\includegraphics[width=0.5\linewidth]{diagrams/decistree.pdf}
\caption{A binary decision tree for $x \vee y$}
\label{fig:decis_tree}
\end{figure}

Figure \ref{fig:decis_tree} shows a decision tree for the disjunction of two variables. The root node represents the disjunction function. The child nodes, or internal nodes, represent variables and the leaf nodes, or terminal nodes represent the outcome of the function. Terminal nodes are labelled either True or False. Given a valuation of X and Y, we can evaluate the function by starting at the root and taking the solid edge if the variable represented by the node is assigned to True in the valuation and taking the dashed edge if the value is assigned to False. 

For example, in figure \ref{fig:decis_tree}, for the valuation $x=True$ and $y=False$, we start at the root node which is labelled x as x is this node's decision variable. X is assigned to True so we follow the solid edge to the next decision node which is labelled y. Y is assigned to false so we take the dashed edge and arrive at a terminal node whos value is 1, meaning that the function evaluates to 1 for this variable valuation.

\subsubsection{Ordered binary decision trees}

If the order in which the variables appear along all paths starting from the root node and ending at a terminal node are the same, then the decision tree is called an ordered decision tree. The decision tree in figure \ref{fig:decis_tree} is ordered.

\subsubsection{Reduced ordered binary decision diagrams}

\begin{figure}[t]
\centering
\includegraphics[width=0.5\linewidth]{diagrams/bdd.pdf}
\caption{A binary decision diagram}
\label{fig:bdd}
\end{figure}

A reduced ordered binary decision diagram (from now on, just BDD) is created by sharing subtrees as much as possible within an ordered binary decision tree.

In particular: 
\begin{itemize}
    \item Terminal nodes with the same label are merged. This means there are only two terminal nodes: True and False.
    \item Internal nodes with the same children are merged.
    \item Nodes with two identical children are removed and all incoming nodes are redirected to the child.
\end{itemize}

Figure \ref{fig:bdd} is an example of a BDD.

\subsubsection{Complement arcs}

\subsubsection{Canonicity}
Given a variable ordering, reduced ordered binary decision diagrams are a canonical representation of a function. This means that given a function $f$ of some set $S$ of variables, another function $g$ that evaluates to the same value for each valuation of the variables in $S$ will be represented by exactly the same BDD.

In practice, one uses a BDD library such as CUDD \cite{cudd} to build and manipulate BDDs. CUDD keeps track of all BDDs and subtrees within the BDDs that have been created and reuses these to ensure that all BDDs remain in reduced form. This, along with canonicity, means that BDD equivalence can be checked in constant time simply by checking pointer equality of the two BDDs.

\subsubsection{Conjunction and disjunction}

BDDs are not usually built as decision trees and then reduced. Instead, they are build from the bottom up, starting with the terminal and variable nodes and combining these using conjunction, disjunction and negation.

Conjunction and disjunction are computed using a straightforward recursive algorithm that wont be given here, but if the reader is interested, more information can be found in \cite{somenzi_bdd}. An important result is that, in the worst case, the procedure runs in time proportional to the product of the sizes of the two input BDDs. Furthermore, the size of the resulting BDD may be equal to the product of the sizes of the two input BDDs in the worst case. A strength of BDDs, however, is that this worst case rarely happens in practice. 

\subsubsection{Function composition and quantification}

Function composition is where a BDD representing some function is substituted for a variable in another BDD.

Given a function $f(x_1,\cdots,x_n)$, we define existential quantification of $f$ with respect to $x_i$ as $\exists x_i. f = f_{x_i} \vee f_{x_i}$ and universal quantification of $f$ with respect to $x_i$ as $\forall x_i. f = f_{x_i} \wedge f_{x_i}$.

Quantifications of the same type commute, so quantification with respect to a set of variables is well defined. 

\subsubsection{Variable ordering}

The number of nodes in a BDD depends drastically on the ordering chosen for the variables. Therefore, the space occupied by the BDDs and the time spent performing operations on them also depends on the variable ordering. This directly affects the performance of game solving algorithms that use BDDs as the symbolic data structure. 

Optimal variable orderings may be found using exact algorithms, but these are prohibitively expensive for BDDs with more than a few nodes. In practice heuristics are used which produce good, but not optimal orderings. One such heuristic is Ruddell's sifting algorithm \cite{sifting}.

The CUDD BDD package performs \emph{dynamic variable ordering}, which means that once the number of BDD nodes the package knows about grows past a certain threshold, the package automatically performs the requested reordering algorithm on all BDDs that exist in the manager. Dynamic variable ordering is critical to the performance of game solving algorithms that utilise BDDs and therefore we always enable it.

\chapter{Driver synthesis as a Game}

\section{Scope}
What goes here?

\section{Formalism}

\subsection{Concurrent games}

Our game formalism for termite makes use of concurrent games. Concurrent games differ from simpler turn based games in that in each state both players get to pick a label and the next state that the game transitions to is some function of both of those labels. Turn based games are a special case of concurrent games where in player~i states the next state is entirely determined by the label played by player~i and the other player's label is ignored. 

It is important to specify which player gets to pick their label first and if the second player gets to have knowledge of the label that the first player picks when choosing their label. In Termite, player~1 (the driver) has to pick first and the environment (device and operating system) gets to pick second with knowledge of the label that the driver picked. This makes the game more difficult to win for the driver.

We call the part of the label that the driver gets to pick $C$ (for controllable) and the part that the environment gets to pick $U$ (for uncontrollable). Our transition relation is defined over the current state $S$, as well as the label $C$ and $U$. Our controllable predecessor becomes:

\begin{equation}
    CPre(X) = \exists C. \forall U. \forall N. TRANS(S, C, U) \rightarrow X'
\end{equation}

We define a special variable that is part of the environment's label called \textsc{Turn}.

\subsection{Device and OS state machines and synchronization}

\subsection{GR(1) based formalism}

As a concrete example, we could create a crude formalism for driver synthesis using only a reachability game. Consider, for example, figure \ref{fig:reach}, which shows the state machine for a game to control a hypothetical network controller. Solid lines indicate controllable transitions and dashed lines indicate uncontrollable transitions. Execution begins in the leftmost state where the OS may initiate a network transfer by choosing the `send' label. The goal of the game is the rightmost state (labelled `G') as this is the point where player 1 has completed the request. So, to win, player 1 (who controls the transitions with solid lines) must ensure that execution of the state machine reaches the goal. 

\begin{figure}[t]
\centering
\includegraphics{diagrams/reachgame.pdf}
\caption{Reachability game for simple network device}
\label{fig:reach}
\end{figure}

The network device has two 8-bit registers, command (abbreviated cmd) and data. Writing 0x01 to the command register starts the transfer, and eventually whatever is in the data register gets written out to the network. Note that the actual sending of the data is an uncontrollable event. 

The correct sequence to win the game, therefore, is to write the data register and then the control register after the OS performs a send request. This takes us to state `S5' where the only move by player 2 is `evt\_send' taking us to the goal. 

If the command register is written first and then the data register there is potential for the environment to play the `evt\_send' label before the data is written, potentially resulting in the wrong data being sent. This is the transition that terminates in the `E' state (for error). The `E' state is a dead end, so it is not possible to reach the goal. 

So, if player 1 takes the top half of the diamond (ie. writes data before command) then it will be guaranteed to reach the goal and the reachability game is winning for player 1. The strategy to reach the goal tells us the sequence of labels the driver must play to get to the goal. In principle, this could be turned into a driver for our simple network device.

\section{\buchi, fairness and GR(1) games}

This simplistic formalism for driver synthesis has several shortcomings that we will deal with in the following sections.

\subsection{We must be able to repeatedly satisfy the OS requests}

Consider a simplified network controller that does not have a command register. Instead, writing to the data register triggers transmission of the byte. However, there are two ways of writing to the data register. One is a standard register write. The other also performs the register write and then schedules a self destruct sequence to happen immediately after the byte is transmitted. The state machine for this device is shown in figure \ref{fig:buchi}. The goal, in this case, is the set ${S3, S5}$ corresponding to the state after completion of the send request. The problem is that, unless you only ever want to send one byte, this goal does not capture the required behavior. One could easily work around this problem by specifying only ${S3}$ as the goal, 

The solution is to modify the objective of the game. Instead of being able to reach the goal once, we want to be able to reach the goal an infinite number of times. Or, equivalently, we want to always be able to reach the goal again. This kind of objective is called a Buchi objective and a game with a Buchi objective is called a Buchi game. 

\begin{figure}[t]
\centering
\includegraphics{diagrams/buchigame.pdf}
\caption{Buchi game for simple network device}
\label{fig:buchi}
\end{figure}

\subsection{We must be able to rule out invalid behaviors not easily expressed with state machines}

Consider a modification of our simplified network device without a self destruct sequence, but with the ability to check that noone is using the communication medium prior to transmitting. The state machine of this device is given in figure \ref{fig:fair}. After the user requests data transmission by writing to the data register, it executes a loop that checks if the medium is free, and if so, it performs the transmission. 

If we pose this as a reachability game with goal state $G$, then the game is not winnable. The device may stay in the loop forever as it is never guaranteed to exit. Such a behavior should not prevent a driver from being synthesized providing that we have good reason to believe that the loop will eventually exit. Looping forever can be seen as a invalid behavior and we want to synthesize a driver for this system providing the invalid behavior does not occur. 

In model checking these behaviors are eliminated with fairness conditions. Fairness conditions are sets of states which we guarantee will eventually be left, which we refer to as unfair states. In the example, the unfair states are the set ${S2, S3}$. The fairness condition says that we will eventually leave the unfair set, and the only way of doing this is through the $evt\_send$ transition, and the game becomes winning.

\begin{figure}[t]
\centering
\includegraphics{diagrams/fairreach.pdf}
\caption{Fair reachability game for simple network device}
\label{fig:fair}
\end{figure}

\section{Game based formalism for drivers}

The combination of fairness and buchi objectives is called a GR(1) objective. Intuitively a GR(1) objective says that we can always reach some goal state provided that we do not get stuck forever in some unfair set of states. We use GR(1) objectives in Termite as we have found that in practice it is sufficient to express our goals.

\chapter{Solving games efficiently}

We have a formalism for the driver synthesis problem as a game. A practical driver synthesis tool using the game formalism must be able to solve and find strategies for these games for real device and operating system specifications in a reasonable amount of time. The principle challenge of this work is creating a synthesis algorithm that scales well enough to handle the large state machines of real device and operating system specifications. 

The straightforward symbolic solver that uses BDDs as the symbolic data structure is remarkably efficient. In fact, it is the current state of the art in reactive synthesis. I will use this as the starting point for my description of Termite's game solver.

I will begin by describing my entry to the reactive synthesis competition in 2014. The solver won the sequential realizability category, the only category in which it was entered.  

\section{Synthesis competition}

\section{Symbolic Solver}
\label{sec:syntcomp}

The starting point for the design of an efficient game solver is the standard BDD-based symbolic algorithm of Section~\ref{sec:back_symbolic_alg}. This standard symbolic algorithm forms the basis of my entry to the Reactive Synthesis competition, named `Simple BDD Solver'. The basic algorithm admits a number of optimisations. As I will show in Section~\ref{sec:syntcomp_eval}, these optimisations significantly improve the performance of the algorithm, but not enough to synthesize even simple device drivers. Yet, this optimised symbolic algorithm is used inside the abstraction-refinement loop of Termite to solve the abstract games produced at every refinement iteration. I therefore describe each of them below.

This section focuses on safety games, as required for the synthesis competition. However, all of the optimisations presented here apply equally to reachability games and to the GR(1) games used in Termite. 

\subsection{Overview}
Following Section~\ref{sec:back_symbolic_alg}, the standard BDD-based symbolic algorithm computes:

\begin{equation}
\label{eqn:mu_syntcomp}
\nu X. Cpre(X \lor \sigma)
\end{equation}

\noindent where $\sigma$ is the safety condition.

The controllable predecessor (Section~\ref{sec:back_cpre}) is specific to the type of game being solved. I give the controllable predecessor for the synthesis competition below as it is simpler than Termite's and is used in the discussion of optimisations below.

\begin{equation}
\label{eqn:cpre_syntcomp}
Cpre(X) = \forall U. \exists C. \forall S'. (\delta(S, U, C, S') \rightarrow X')
\end{equation}

\noindent where $U$ is the set of valuations of uncontrollable inputs, $C$ is the set of valuations of controllable inputs, and $S$ is the set of valuations of state variables. Given a set $X$, $X'$ denotes the next state copy of $X$ and $\delta$ is the transition relation.

\subsection{Optimisations}
\label{sec:syntcomp_optimisations}

The optimisations that I have used, in approximate order of importance, are:
\begin{itemize}
    \item Dynamic variable reordering using the sifting algorithm \cite{Rudell_1993}
    \item Dereference unused BDDs as soon as possible
    \item Partitioned transition relations \cite{Burch_91} and direct substitution with \textsc{Cudd\_VectorCompose}
    \item Simultaneous conjunction and quantification with \textsc{Cudd\_BddAndAbstract}
    \item Terminate early where possible
\end{itemize}

The optimised algorithm (Algorithm~\ref{alg:syntcomp}) and controllable predecessor (Algorithm~\ref{alg:syntcomp_cpre}) are given in the implementation section (Section~\ref{sec:syntcomp_impl}).

\subsubsection{Dynamic variable ordering}
I do not try to find a good static variable ordering at the start and instead rely on the sifting algorithm provided by the CUDD package for finding good variable orderings dynamically. In my experience, the sifting algorithm provides the best tradeoff between the quality of the resulting ordering and time taken to find it. I enable sifting at the start so that it is active during both compilation and solving. I did not modify any of the default parameters to the sifting algorithm.

\subsubsection{Dereferencing dead BDDs}
I dereference BDDs that are no longer needed as soon as possible. Each live BDD node is processed during reordering and counted when the algorithm checks to see if the total BDD size in the manager is reduced. These unused BDDs should not count toward the total node count that is used to evaluate an ordering, and, the time that is spent reordering them is wasted. Furthermore, these unused BDDs occupy memory and, if not dereferenced, may use up all of the memory available on the system.

\subsubsection{Partitioned transition relations}
\label{sec:syntcomp_partitioned}
I do not compute the transition relation as a monolithic BDD defined over current state, input variables and next state. This would likely be very large and slow down the algorithm considerably. Instead, I keep it in a conjunctively partitioned form with one partition for each next state variable as described below. 

This optimisation is sound because the next state value of any state variable depends only on the current and input variables and not any other next state variables. Furthermore, the next state value of any state variable is deterministic. This means that it can be represented directly as a function of the current state and input variables. I use a BDD defined over current state and input variables to represent this function. The transition relation becomes a list of BDDs, one for each state variable, each of which only depends on current state and input variables. 

To compute the implication, $\forall S'. (\delta(S, U, C, S') \rightarrow X')$, I use the CUDD function \textsc{Cudd\_VectorCompose} as shown in algorithm \ref{alg:syntcomp_cpre} on line~\ref{l:vc} to substitute each update function into X. This avoids building the monolithic transition relation and, importantly, it avoids having to ever declare a next state copy of each state variable in the BDD manager. 

\subsubsection{Simultaneous conjunction and quantification}
I perform simultaneous conjunction and existential quantification wherever possible. CUDD implements a composite operation that performs quantification and conjunction simultaneously called \textsc{Cudd\_BddAndAbstract}. Given two BDDs $x$ and $y$ and a set of variables $v$, it computes $\exists v. x \land y$ as a single operation without explicitly building the conjunction $x \land y$. I perform this optimisation on line~\ref{l:andabs} of Algorithm~\ref{alg:syntcomp_cpre}. As this avoids building the potentially large BDD representing the conjunction it saves both memory and time.

\subsubsection{Early termination}
I terminate early when possible. As I am computing a greatest fixed point, I start with the universal set and progressively shrink it to find the winning region. Each time I shrink the winning region, I check that is it still a superset of the initial set. If it is not, I know there is no way player~1 can win as the winning set only shrinks as the algorithm progresses. I use the function \textsc{Cudd\_bddLeq} on line~\ref{l:leq} of Algorithm~\ref{alg:syntcomp} for this purpose as it efficiently checks that one BDD implies another without constructing the BDD of the implication.

\subsection{Implementation}
\label{sec:syntcomp_impl}
I developed two implementations that incorporate all of the above optimisations. The first implementation is used by Termite inside the abstraction refinement loop in order to solve abstract games (Section~\ref{sec:abs_ref_pred_abs}). The second solver was developed for the reactive synthesis competition. 

The Reactive Synthesis Competition~\cite{syntcomp_arxiv} is a competition for reactive synthesis tools inspired by competitions in other fields such as the SAT competition~\cite{satcomp} and the Hardware Model Checking Competition~\cite{hwmcc}. The competition had four tracks: sequential realisability, parallel realisability, sequential synthesis and parallel synthesis. The tools were required to solve safety games given in an extension of the AIGER format \cite{aiger}. Entrants in the synthesis categories were required to produce an implementation of a controller that enforced the safety condition, also given in extended AIGER format. Entrants in the realisability category were only required to determine if the safety game was winnable, not to produce a strategy.

Two different implementations were required because the synthesis competition requires a different controllable predecessor to driver synthesis (Section~\ref{sec:termite_cpre}), and, modifications to allow for predicate abstraction were necessary for Termite (Section~\ref{sec:abs_ref_pred_abs}).

The synthesis competition solver is written in the Haskell functional programming language. It uses the CUDD \cite{cudd} package for binary decision diagram manipulation and the Attoparsec Haskell package for fast parsing. Altogether, the solver, AIGER parser, compiler and command line argument parser are just over 300 lines of code. The code is available online at: \path{https://github.com/adamwalker/syntcomp}.

\begin{algorithm}
\caption{Syntcomp Controllable predecessor}
\label{alg:syntcomp_cpre}

\begin{algorithmic}[1]

\Function{CPre}{$C, U, \sigma, target$}

\State $substituted \gets \Call{Cudd\_VectorCompose}{target, \delta}$ \label{l:vc}
    \State $safeSub     \gets \Call{Cudd\_bddAndAbstract}{C, \sigma, substituted}$ \label{l:andabs}
    \State $winning     \gets \Call{Cudd\_bddUnivAbstract}{U, safeSub}$
    \State \Return $winning$

\EndFunction

\end{algorithmic}
\end{algorithm}

\begin{algorithm}
\caption{Syntcomp symbolic solver}
\label{alg:syntcomp}

\begin{algorithmic}[1]

\Function{Solve}{$\sigma, init, \delta, C, U$}

    \State $win \gets \Call{Cudd\_ReadLogicOne}{}$
    \Loop
        \State $res' \gets \Call{CPre}{C, U, \sigma, res \land \sigma}$
        \State $win  \gets \Call{Cudd\_bddLeq}{init, res'}$ \label{l:leq}
        \If{$\neg win$} 
            \State \Return $False$
        \EndIf
        \If{$res = res'$} 
            \State \Return $True$
        \EndIf
        \State $res \gets res'$
    \EndLoop

\EndFunction

\end{algorithmic}
\end{algorithm}

The optimised algorithm is given in Algorithm~\ref{alg:syntcomp}. It calls the optimised controllable predecessor suitable for the synthesis competition defined in Algorithm~\ref{alg:syntcomp_cpre}.

\subsection{Evaluation of Optimisations}
\label{sec:syntcomp_eval}

\begin{sidewaystable}
    \small
    \center
    \begin{tabular}{|l|S[table-format=2.2]|S[table-format=2.2]|S[table-format=2.2]|S[table-format=2.2]|S[table-format=2.2]|S[table-format=2.2]|}
        \hline
        \multirow{2}{*}{Benchmark} & \multicolumn{6}{c|}{Optimisation} \\ \cline{2-7} & 
        \multicolumn{1}{c|}{None}   & \multicolumn{1}{c|}{+ Reord.} & \multicolumn{1}{c|}{+ Deref.} & \multicolumn{1}{c|}{+ Partitioned} & \multicolumn{1}{c|}{+ Simult. abs.} & \multicolumn{1}{c|}{+ Early term.} \\
        \hline

        amba02\_new\_08n\_unreal.aag    & 34.73  & 37.00    & 1.15     & 0.63          & 0.28           & 0.22          \\
        amba02\_new\_08n\_unreal\_o.aag & 34.25  & 36.99    & 0.57     & 0.69          & 0.26           & 0.26          \\
        amba02\_new\_09n.aag            & 28.50  & 28.37    & 0.47     & 0.37          & 0.20           & 0.24          \\
        amba02\_new\_09n\_o.aag         & 28.20  & 30.73    & 0.62     & 0.40          & 0.24           & 0.28          \\
        amba03\_new\_08n\_unreal.aag    & OOM    & 391.59   & 1.94     & 0.88          & 0.94           & 0.95          \\
        amba03\_new\_08n\_unreal\_o.aag & OOM    & 426.67   & 1.76     & 1.45          & 0.82           & 0.47          \\
        amba03\_new\_09n.aag            & OOM    & 452.90   & 3.98     & 1.78          & 0.89           & 1.13          \\
        amba03\_new\_09n\_o.aag         & OOM    & 453.70   & 5.51     & 2.05          & 1.13           & 1.13          \\
        amba04\_new\_24n\_unreal.aag    & OOM    & OOM      & 28.40    & 5.48          & 4.91           & 2.54          \\
        amba04\_new\_24n\_unreal\_o.aag & OOM    & OOM      & 11.11    & 15.73         & 3.33           & 4.65          \\
        amba04\_new\_25n.aag            & OOM    & OOM      & 11.60    & 7.89          & 4.25           & 4.48          \\
        amba04\_new\_25n\_o.aag         & OOM    & OOM      & 9.12     & 7.50          & 5.41           & 7.31          \\
        amba05\_new\_16n\_unreal.aag    & OOM    & OOM      & 11.86    & 12.93         & 6.97           & 5.13          \\
        amba05\_new\_16n\_unreal\_o.aag & OOM    & OOM      & 17.93    & 11.79         & 6.28           & 5.56          \\
        amba05\_new\_17n.aag            & OOM    & OOM      & 31.36    & 5.47          & 7.37           & 6.53          \\
        amba05\_new\_17n\_o.aag         & OOM    & OOM      & 15.90    & 10.61         & 5.00           & 3.87          \\
        amba06\_new\_20n\_unreal.aag    & OOM    & OOM      & 50.89    & 32.04         & 6.15           & 5.44          \\
        amba06\_new\_20n\_unreal\_o.aag & OOM    & OOM      & 55.96    & 31.94         & 11.04          & 7.08          \\
        amba06\_new\_21n.aag            & OOM    & OOM      & 49.15    & 27.95         & 11.07          & 11.98         \\
        amba06\_new\_21n\_o.aag         & OOM    & OOM      & 21.12    & 21.85         & 12.61          & 27.48         \\
        amba07\_new\_24n\_unreal.aag    & OOM    & OOM      & 67.95    & 21.37         & 19.68          & 17.76         \\
        amba07\_new\_24n\_unreal\_o.aag & OOM    & OOM      & 45.36    & 91.67         & 20.14          & 17.94         \\
        amba07\_new\_25n.aag            & OOM    & OOM      & 127.04   & 53.93         & 15.90          & 18.38         \\
        amba07\_new\_25n\_o.aag         & OOM    & OOM      & 51.00    & 51.10         & 24.50          & 18.33         \\

        \hline
    \end{tabular}
    \caption{Runtimes (in seconds) of the symbolic solver with the optimisation of Section~\ref{sec:syntcomp_optimisations} progressively enabled}
    \label{tab:syntcomp_optimisations}
\end{sidewaystable}

Table~\ref{tab:syntcomp_optimisations} gives the runtimes, in seconds, of the symbolic solver on a selection of benchmarks from the 2014 synthesis competition. Benchmarks from the AMBA category were used. These benchmarks model an arbiter for the AMBA AHB bus, based on an industrial specification by ARM. The benchmarks were run on an Intel Haswell laptop with 4 gigabytes of ram and a processor speed of 1.6 GHz. Benchmarks containing the text `unreal' in the name are unsatisfiable and all others are satisfiable. If the runtime is listed as OOM, it means that the solver ran out of memory.

The first column gives the runtimes for the symbolic solver without optimisations. Most entries are failures due to out of memory conditions. I have enabled the optimisations progressively in subsequent columns. In the next two columns, dynamic variable reordering and BDD dereferencing are enabled. It is clear that these optimisations are crucial to the performance of the algorithm. In subsequent columns, I enable partitioned transition relations and simultaneous conjunction and quantification. These optimisations successively improve performance on nearly all benchmarks. Finally, in the last column, I enable early termination. This does not improve performance on many of the benchmarks. Early termination can only help when the benchmark is unsatisfiable. When the benchmark is satisfiable it only imposes the additional work of checking containment in the initial set. This optimisation does, however, improve performance substantially on some of the unsatisfiable benchmarks. It is unclear whether it is worthwhile.

The benchmark runner is available online at: \url{https://github.com/adamwalker/syntcomp-benchmark}. The set of benchmarks is available online at \url{https://syntcompdb.iaik.tugraz.at/static/SyntComp.tar.gz}.

\subsection{Conclusion}
Simple BDD Solver performed well compared to the other solvers entered in the reactive synthesis competition as a result of the optimisations presented above. It failed when the BDDs representing the winning sets, or the intermediate BDDs in the controllable predecessor computation grew too large. Additionally, it failed when a large number of iterations was required to determine the outcome, such as the benchmarks with a large counter. 



\section{Abstraction}
An abstraction is a simplification of the original transition system. An abstraction is used when the game is too large to be solved. Ideally, an abstraction is both small enough to be solved and detailed enough to gain some additional information about the properties of the system. 

One common use of an abstraction is in an abstraction-refinement loop. In an abstraction refinement loop, an initial simple abstraction is found and is solved. Then, the results of the abstraction are used to refine the abstraction, ie. to build another system model that contains slightly more detail than the original abstraction. This is repeated in a loop until the original game is solved. It is often possible to solve the original game with a far less detailed abstraction that the original system. 

Finding an abstraction that is simultaneously small and useful for making progress in solving the game is a difficult task and is what will be dealt with in the following sections. We start with earlier work on three valued abstraction refinement. 

\section{Three valued abstraction refinement}

The idea is that given an abstraction, we classify states into one of three categories: winning, losing, and unknown. If we discover that the entire initial set is winning, we know that the original game is winning and we can terminate. Dually, if we discover any initial state that is losing, we know that the entire initial set can never be winning, hence the game is losing and we can terminate. 

At termination, either 
\begin{itemize}
\item all of the initial states are classified as winning (but the other states need not be classified), or
\item one of the initial states is classified as losing (again, no other states need to be classified)
\end{itemize}

This additional imprecision often allows us to use a less precise abstraction compared to the original algorithm where all states are exactly classified. 

We need to describe what an abstraction actually is and how we use this abstraction to classify states. 

\subsection{Abstraction}

An abstraction of a game structure $G$ is a tuple $\langle V, 
\concrete{}\rangle$, where $V$ is a finite set of abstract states 
and $\concrete{} : V \rightarrow 2^S $ is the \emph{concretisation 
function}, which takes an abstract state and returns the possibly 
empty set of concrete states that the abstract state corresponds 
to.  We require that $\bigcup_{v\in V}\concrete{v} = S$ and         
$\concrete{v_1}\cap \concrete{v_2} = \emptyset$ for any $v_1$ and 
$v_2$, $v_1 \neq v_2$. In the case when $\concrete{v} = \emptyset$ 
the abstract state $v$ is said to be \emph{inconsistent}.  We 
extend the $\concrete{}$ operator to sets of abstract states.  For 
$U\subseteq V$: $\concrete{U} = \bigcup_{u\in U}\concrete{u}$.

\subsection{Algorithm}
In this section we present a modified version of the three-valued 
abstraction refinement technique of de~Alfaro and 
Roy~\cite{Alfaro_Roy_07}.  To simplify the presentation, we focus 
on solving reachability games.  De~Alfaro and Roy present an 
extension of their method to arbitrary $\omega$-regular games.  
This extension is directly applicable to the version of the 
algorithm presented here.

%Given an abstraction $\langle V, \concrete{}\rangle$ of a game 
%$G$, the three-valued abstraction refinement scheme computes 
%over- and under-approximations of the winning region of the game.  
%If necessary, the abstraction is refined in order to narrow down 
%the gap between the two.
We start with defining two versions of the abstraction operator: 
the \emph{may-abstraction} $\abstractm{}$ and the 
\emph{must-abstraction} $\abstractM{}$. For a set of concrete 
states $T \subseteq S$:
$\abstractm{T} = \{v\in V\mid \concrete{v} \cap T \neq 
\emptyset\}$, $\abstractM{T} = \{v\in V\mid \concrete{v} \subseteq 
T \}$.
We say that abstraction is \emph{precise} for a set $T\subseteq S$ 
if $\concrete{(\abstractm{T})} = \concrete{(\abstractM{T})}$.

Next, we define may and must versions of the abstract controllable 
predecessor operator:
\begin{equation}
    \small
    Cpre_i^m(U) = \abstractm{Cpre_i(\concrete{U})},~~
    Cpre_i^M(U) = \abstractM{Cpre_i(\concrete{U})}
\label{e:cpremM}
\end{equation}
These operators have the property:
$\concrete{Cpre_i^M(U)} \subseteq Cpre_i(\concrete{U}) \subseteq \concrete{Cpre_i^m(U)},$ and hence
$\concrete{\reach(\abstractM{T}, Cpre_i^M)} \subseteq \reach(T, Cpre_i) \subseteq \concrete{\reach(\abstractm{T}, Cpre_i^m)}.$

The abstract $Cpre_i^m$ and $Cpre_i^M$ operators are defined in 
terms of the concrete controllable predecessor $Cpre$. As these 
may not be possible to compute efficiently in practice, we 
introduce approximate versions, $Cpre_i^{m+}$ and $Cpre_i^{M-}$, 
such that for all $U\subseteq V$: $\concrete{Cpre_i^m(U)} 
\subseteq \concrete{Cpre_i^{m+}(U)}$ and 
$\concrete{Cpre_i^{M-}(U)}
\subseteq \concrete{Cpre_i^M(U)}.$  The definition of 
$Cpre_i^{m+}$ and $Cpre_i^{M-}$ is determined by each particular 
instantiation of the abstraction refinement scheme.  We present 
our version of these operators in Section~\ref{s:cpre}.  

Figure~\ref{f:reach} illustrates the main idea of our approach, 
which is presented in algorithm~\ref{alg:generic}.  At every 
iteration, the algorithm computes the must-winning set $W^M$ that 
underapproximates, and the may-winning set $W^m$ that 
overapproximates the true winning set (lines~2--3).  The algorithm 
terminates if the must-winning set contains the entire initial set 
or the may-winning set has shrunk beyond the initial set 
(lines~4--5).  Otherwise, the algorithm refines the abstraction in 
a way that expands the must-winning set.
%To this end we identify a set of states from 
%which player~1 can force the game to $W^M$ in one step.  Since all 
%states in $W^M$ are must-winning, the new state will be 
%must-winning as well.  Furthermore, due to the nature of the 
%reachability game, all winning states can eventually be discovered 
%in this way.  

\begin{algorithm}[t]
\caption{Three-valued abstraction refinement for games.}
\label{alg:generic}

\begin{algorithmic}[1]

% \Function{Solve}{$transitionRelation$, $goal$}
    \Statex {\bf Input:} A game structure $G = \langle S, L, I, \tau_1, \tau_2, \delta \rangle$, a set 
    of target states $T\subseteq S$, and an initial abstraction $\alpha=\langle V, \concrete{}, Cpre_1^{m+}, Cpre_1^{M-} \rangle$
    that is precise for $T$, $I$, and $\tau_i$.

    \Statex {\bf Output:} {\it Yes} if $I \subseteq \reach(T, Cpre_1)$, and {\it No} otherwise.

    \Loop
        \State $W^M \gets \reach(\abstractM{T}, Cpre_1^{M-})$
        \State $W^m \gets \reach(\abstractm{T}, Cpre_1^{m+})$
        \If{$\abstractM{I} \subseteq W^M$} 
            \State\Return Yes
        \ElsIf{$\abstractM{I} \nsubseteq W^m$} 
            \State\Return No
        \Else       
            \State $refined \gets \Call{refineCpre}{W^M}$
            \If {$(\neg refined)$}
                \State$\Call{refineAbstraction}{W^M}$
            \EndIf
        \EndIf
    \EndLoop
\end{algorithmic}
\end{algorithm}

The key observation behind the refinement procedure is that 
candidate winning states can be found at the \emph{may-must 
boundary} of the game, i.e., the set $Cpre_1^{m+}(W^M)\setminus 
W^M$, of all may-predecessors of the must-winning set.  The 
boundary consists of three regions shown in Figure~\ref{f:reach}: 
(1) $Cpre_1^{M}(W_M)\setminus W_M$, (2) $Cpre_1^m(W^M)\setminus 
Cpre_1^{M}(W^M)$, and (3) $Cpre_1^{m+}(W^M)\setminus 
Cpre_1^{m}(W^M)$.  The first and the third regions can be shrunk 
by increasing the precision of the $Cpre^{M-}$ and $Cpre^{m+}$ 
operators respectively.  The second region can only be shrunk by 
refining the abstraction itself, i.e., partitioning abstract 
states into smaller regions.

These two types of refinement are performed in lines~7 and~8 of 
the algorithm.  The \textsc{refineCpre} function computes a more 
precise version of the controllable predecessor operators.  It 
returns $false$ iff no such refinement is possible, i.e., 
$Cpre^{M}(W_M)=Cpre^{M-}(W_M)$ and $Cpre^{m+}(W^M)=Cpre^{m}(W^M)$.  
The \textsc{refineAbstraction} function refines the abstract state 
space in a way that expands the set $Cpre^M(W^M)$ with at least 
one new abstract state.  
%The repeated application of \textsc{refineCpre} 
%and \textsc{refineAbstraction} has the effect of eventually 
%extending the must-winning region $W^M$ with new states, which 
%guarantees termination of the algorithm.

Algorithm~\ref{alg:generic} differs from \cite{Alfaro_Roy_07}
in that it uses an additional type of refinement which refines the 
controllable predecessor operators without changing the abstract
state space.

%Use reachability
%The 3 valued abstraction refinement paper uses an abstraction that is precise wrt the initial state. We dont need this. What is precise?
%Iteratively classify states in may but not must as winning or not.
%Describe it

\subsection{An improved symbolic implementation}

From an email to the german guys:

It's a simple abstraction-refinement loop. At the start, the safety condition is compiled to a BDD. In the master branch of my solver, I then compile update functions for each state variable. In the "untracked" branch, I only compile update functions for each state variable that occurs in the safety condition. However, these update functions may depend on additional state variables. Instead of compiling update functions for these additional state variables, I make them input variables. I refer to these as untracked variables. The state space of the game is now much smaller. The player (ie. not the environement) is free to choose the values of these inputs. This makes the game easier to win for the player.

The simplified game is then solved. If the game is lost, then the player certainly loses. If the player wins, it may because the abstraction made it easer to win the game. To detect this, we do the controllable predecessor one more time, but dont quantify out the untracked variables. We pick a small losing <state, untracked> cube and change the untracked variables to state variables by computing their update functions, and solve the game again. If no such cube exists, the game is winning.

On each refinement iteration, the winning region can only shrink, so we reuse the winning region from solving the last game on the next iteration. Also, if we were solving a reachability game, we would let the environment pick the valuations of the untracked variables and the winning sets would only grow.

In practice, this works very well for the games I am solving to synthesize controllers for computer hardware. It does not work so well on the synthesis competition benchmarks because most of the state is needed to determine whether the game is winning, though, for some reason, it can create simpler abstractions for a lot of the unrealizable benchmarks. 

\subsection{Extension to GR(1) games and correctness proof}

from the FMCAD paper

The algorithm begins in the same way as the reachability algorithm. We solve the game with the current abstraction to find $W^m$ and $W^M$ and terminate if these allow us to determine the outcome. We then refine and solve again. Again, as in the reachability case, we aim to grow $W^M$ and shrink $W^m$ so that, if we do not terminate early, they will eventually become the same set. This guarantees termination through one of the if conditions. Predicate promotion and consistency refinement are the same as before but operate on different boundary states. We describe the algorithm to find these boundaries. 

Suppose the specification formula is of the form $\nu X. \psi$, ie. it has a greatest fixed point at its outermost level. We have already calculated the may winning region and we denote this $X^m$ as it is the final value that the $X$ variable takes when solving the may game. We attempt to directly shrink $X^m$ by reconsidering the last application of $Cpre^m$ that yielded $X^m$ and looking for refinements that cause some of the may winning states found by this last iteration to become losing. This amounts to checking the $X^m$-lose boundary for additional losing states, in the same way safety games are refined, which also happens to be specified with a greatest fixed point.

We redo the last $CPre$ application as follows. We evaluate $\phi(X, Y, Z, ...)$ with each fixed-point-quantifier variable substituted as $X^m$, ie. $X=X^m$, $Y=X^m$, ... as these are the values that the fixed point variables had in the last iteration when the game was solved. This happens because the $mu$-calculus formula is in prefix normal form. We then use this value as the target and refine states and consistency relations as described previously. We re-solve if we succeed.

Making refinements only as described above does not guarantee that eventually $W^m = W^M$. We refine recursively as follows. We define a new objective: $\mu Y., ..., cpre(\phi(X=X^m, Y, ...))$, ie. we drop the $\nu X.$ quantifier and replace $X$ by $X^m$ and refine recursively with this. Note that $Y^m \neq Y^M$ as otherwise $X^m$ would equal $X^M$ and we would have terminated. Conceptually, we are trying to either grow $Y^M$ to $X^m$ ($=Y^m$) through repeated refinement, proving that $X^M = X^m$ (and terminating with an answer somewhere along the way), or find a reason why $Y^M$ does not equal $X^m$ (finding this is equivalent to shrinking $Y^m$, and hence $X^m$) and continue, having achieved our goal of bringing $X^m$ closer to $X^M$. One of the two outcomes (growing $Y^M$ to $X^m$ or shrinking $Y^m = X^m$) must happen because they are not equal initially and (by structual induction on the $mu$-calculus formula, assuming the algorithm is correct for shorter formulas, with safety and reachability as the base cases) must meet somewhere in the middle.

Note, that any refinements found in some step would have been found in a subsequent step had that step been skipped. We find that giving priority to the outermost fixed point results in better abstractions and as refinements for outer fixed points are cheaper to compute it makes sense to prioritise them.

Every recursive call drops one fixed point quantifier, so eventually we reach a formula of the form $Q X. \phi(X)$ where $X$ is either $\mu$ or $\nu$ and proceed as in the safety or reachability case. Termination is guaranteed for these, and, by induction for the rest of the specification.

\section{Predicate abstraction}
\subsection{Motivation}
\subsection{Algorithm}

\chapter{User guided synthesis}

\section{Specifications}
\section{Heuristic code generation}
\section{User guided code generation}

\section{Counterexample guided debugging}
\section{Limitations}

\section{A realistic example}

\section{Case studies}

\chapter{Appendix}

\end{document}
